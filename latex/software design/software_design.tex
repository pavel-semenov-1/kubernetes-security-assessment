\chapter{Software design} \label{chap:softwaredesign}

In our work, we are implementing a system that can recognize the following astronomical objects present in the image: 

\begin{itemize}
    \item point source
    \item streak source
    \item streak source that is cut due to being on the edge of the image
    \item elliptical galaxy
    \item cosmic ray
    \item hot pixel
\end{itemize}

As mentioned before, the task of object recognition includes object localization and classification. Because we are using types of astronomical objects, that were not discussed in approaches in \ref{sec:spacerecognition}, we first want to access the possibility of using a neural network to correctly classify our objects into classes. For this reason, we omitted the object localization task. 

The designed network is thoroughly trained on a large amount of synthetic data. Data from real observations are later used to fine-tune the model. The performance of our network is also compared to the state-of-the-art ResNet-18 model. 


\section{Input data} \label{sec:inputdata}

Input data to our network are images in a form of FITS files. Each image contains only one astronomical object as depicted in the Figure \ref{fig:fitsreal3}. The size of the image is 50x50 pixels, as it was mentioned in \cite{Andreon2000} that it is the optimal size of the image for one object to be classified. 
Pixel values of the image are stored in 16 bits, which means that values range from 0 to 65 535. Although the images are in FITS format, we are only making use of the data block and the header is empty. 

In our work, we are using synthetic images as well as real ones. 
However, real images provided by the telescope in AGO have a resolution of 1024x1024 and capture the whole starfield. For us to be able to use these images, we needed to manually cut out 50x50 windows from them with just one astronomical object present. Some examples of such cutouts are shown in the Figure \ref{fig:fitsreal3}.  


\begin{figure}[!h]
   \centering
    \begin{subfigure}{.2\textwidth}
        \centering
        \includegraphics[width=\textwidth]{images/point.png}
        \label{fig:fitsreal3a}
        \caption{Point.}
    \end{subfigure}
    %\hfill
    \begin{subfigure}{.2\textwidth}
        \centering
        \includegraphics[width=\textwidth]{images/line2.png}
        \label{fig:fitsreal3b}
        \caption{Streak.}
    \end{subfigure}
    %\hfill
    \begin{subfigure}{.2\textwidth}
        \centering
        \includegraphics[width=\textwidth]{images/cutline.png}
        \label{fig:fitsreal3c}
        \caption{Cut Streak.}
    \end{subfigure}
    %\hfill
    
    \vspace*{4mm}
    
    \begin{subfigure}{.2\textwidth}
        \centering
        \includegraphics[width=\textwidth]{images/cosmicray.png}
        \label{fig:fitsreal3d}
        \caption{Cosmic ray.}
    \end{subfigure}
    %\hfill
    \begin{subfigure}{.2\textwidth}
        \centering
        \includegraphics[width=\textwidth]{images/hotpixel.png}
        \label{fig:fitsreal3e}
        \caption{Hot pixel.}
    \end{subfigure}
    %\hfill
    \begin{subfigure}{.2\textwidth}
        \centering
        \includegraphics[width=\textwidth]{images/galaxy.png}
        \label{fig:fitsreal3f}
        \caption{Galaxy.}
    \end{subfigure}
    %\hfill
    \caption{Examples of 50x50 pixel cutouts from real images.}
    \label{fig:fitsreal3}
\end{figure}

\section{The network}

For the purpose of object recognition, we are deploying a  convolutional neural network, specifically designed for this thesis. During the training process, the network is fed images along with labels of the class. %To achieve a robust and balanced training dataset, we are generating synthetic images using starGen. Real images are later used to fine-tune the model. 
The validation process is used for determining the best-performing parameters of the network. The network is validated using both synthetic and real images.
After establishing the configuration of the model, the system is tested on real data.

\subsection{The architecture}

When designing the architecture of our network, we took inspiration from LeNet \cite{lenet5}. The model was designed to classify grey-scale images of handwritten digits, where the main indicators of each class are its morphological features. Images in our work are very similar and the difference between each class also relies on the object's morphological features, such as shape, density, etc. 

% prepisat este
The dimensions of the input data to the LeNet network are not consistent with the size of our data. As for this, the topology of the LeNet5 network needed some modifications, to be able to fit our input data. Some changes were made regarding the number of layers and the dimensions of kernels in convolutional layers. 

The topology of network is demonstrated in the Figure \ref{img:arch0} and consists of following layers: 

\begin{enumerate}
    \item \textbf{Layer C1} \\
    The first convolution layer consists of 6 kernels of size 5x5. The convolution operation is performed using a stride of 1 with zero padding on the image. The input to the layer is the grey-scale image of 50x50x1 and the output feature maps are of size 46x46, which makes the output volume 46x46x6. The layer has a total of 156 learnable parameters, including 150 weights and 6 biases. 
    
    \item \textbf{Layer S2} \\
    Subsampling layer that uses average pooling with a kernel of 2x2 size and stride of 2. The input volume to the layer is 46x46x6, and the pooling layer reduces the spatial size by a factor of 2, while the depth stays the same. This results in an output volume of 23x23x6.
    
    \item \textbf{Layer C3} \\
    The second convolution layer with 16 kernels of size 4x4, moving with a stride of 1 and zero padding on the input data. The output consists of 16 feature maps of size 20x20. The layer introduces 96 weights and 1 bias per filter, which is a total of 1552 learnable parameters.
    
    \item \textbf{Layer S4} \\
    The structure of the subsampling layer is the same as the layer S2. The input to the layer is a volume of 20x20x16 which makes the output 10x10x16.
    
    \item \textbf{Layer C5} \\
    The third convolution layer with 32 kernels of size 5x5. Again moving with a stride of 1 and zero padding. The layer outputs the volume 6x6x32. It has a total of 12 832 learnable parameters, with 400 weights and 1 bias per filter. 
    
    \item \textbf{Layer S6} \\
    The last subsampling layer, with the same structure as the previous ones. The layer receives the input volume of 6x6x32 and outputs 32 feature maps of size 3x3.
    
    \item \textbf{Layer C7} \\
    The last convolution layer consists of 120 3x3 convolution kernels, moving with a stride of 1 and zero padding on the input data. As the dimensions of the input data are the same as the convolution kernel, the output is one dimensional tensor with a length of 120. The layer has 289 parameters per filter, which results in a total of 34 680 learnable parameters. 
    
    \item \textbf{Layer F8} \\
    The first fully connected layer contains 84 neurons. The input to the layer is a tensor with a length of 120, or we can also imagine it as 120 neurons. Each neuron from the input is connected to each neuron in the layer. This makes a total of 10 080 connections, where each connection introduces one learnable weight. Each neuron in the layer also contains one bias value. This gives a total of 10 164 learnable parameters.
    
    \item \textbf{Layer F9} \\
    The second fully connected layer where the number of neurons is the number of classes. In the case of LeNet5, it was 10 but in our case, it's 6 neurons. Connecting with the previous layer with 84 neurons creates 504 weights and 6 biases, for a total of 510 learnable parameters. 
\end{enumerate}

The network is made up of 9 layers and has a total of 59 894 learnable parameters. It is standard practice for the topology to consist of alternating convolutional and pooling layers, followed by fully-connected layers at the end. 

Important to mention, that after each convolutional and fully-connected layer, non-linear activation is applied to the data. The specific choice of activation function is not mentioned here as it will be a hyperparameter determined by validation of the network. 


\begin{figure}[h]
    \centering
    \includegraphics[width=\textwidth]{images/architectureCNN2.png}
    \caption[The proposed architecture of the neural network]
    {The proposed architecture of the neural network. 
    For each convolutional layer (denoted as “C”) the information about the kernel and size of output feature maps is defined (N@MxM, where N is the number of filters used and MxM is the spatial size of the feature maps). 
    Subsampling layers ("S") always operate with a 2x2 kernel and reduce only the spatial size of the feature maps. The output information is defined in the same way as with convolutional layers. 
    The last two layers of the network are fully connected ("F") and specify the number of neurons in the layer. The last fully-connected layer is denoted as the output layer.}
    \label{img:arch0}
\end{figure}

\subsection{Parameters} \label{sec:parametersNetwork}

% softmax na konci
% cross entropy na pocitanie gradientu
% early stopping pre val loss aj accuracy
% parametre optimizeru
% scheduler pre LR
% regularization, dropout
% augumentation
% activation function


During the training, images fed to the network come in batches, whose size is determined by parameter \textit{batch size}. Each batch of images is passed through the layers of the network. After each convolution layer, \textit{activation function} is applied to the output. Before the image is passed into the fully-connected layer, it must be flattened into a one-dimensional vector. The last fully-connected layer outputs the class scores of the image. \textit{Softmax activation} is applied to class scores to normalize them into class probabilities, where each value represents how sure our network is that the input is the specific class. A useful feature of softmax activation is that adding probability values for each class together outputs 1. 

\textit{Loss function} is used to measure the performance of the network. It consists of two components: data loss and regularization loss. Data loss calculates the quality of the prediction compared to ground truth labels, for which we use \textit{cross-entropy loss}. \textit{Regularization loss} is only the function of weights, and its goal is to penalize large weights.  

During backpropagation, \textit{optimizer} is used to update the weights of the network to minimize the loss function. The optimizer uses the gradient of the loss function to determine the direction toward the local minima. The size of the step we take in that direction is defined by parameter \textit{learning rate}.  Weights are updated after each pass of the batch through the network - one iteration. When all training data has passed through the network through batches, one \textit{epoch} has passed.

After each epoch, \textit{validation data} is used to determine the performance of our network. Validation loss is computed but it‘s not used to update weights during backpropagation. During the training process, validation loss is most commonly used to adjust the learning rate through \textit{scheduler}. After the network is trained, the performance on the validation data is key to adjusting the hyperparameters of the network to achieve better results. 

The network is trained for a certain number of epochs, which is a hyperparameter defined by the user. However, to avoid overfitting of data, \textit{early stopping algorithm} is deployed, which stops the process of training if needed.  

\subsection{Activation function}

The activation function introduces non-linearity into the neural networks. In our model, it is applied after each convolutional and fully-connected layer and it is the same function in all layers. The baseline model was trained using multiple activation functions with various parameters as shown in the Table \ref{tab:actfuncs}. It shows the validation loss and accuracy of the trained model in the early stopping checkpoint. 

{\renewcommand{\arraystretch}{1.4}
\begin{table}[h]
\centering
\begin{tabular}{|l|cc|cc|}
\hline
\multicolumn{1}{|c|}{\multirow{2}{*}{\textbf{Activation function}}} & \multicolumn{2}{c|}{\textbf{Validation loss}} & \multicolumn{2}{c|}{\textbf{Validation accuracy (\%)}} \\ \cline{2-5} 
\multicolumn{1}{|c|}{} & \multicolumn{1}{c|}{real data} & synthetic data & \multicolumn{1}{c|}{real data} & synthetic data \\ \hline
\textit{sigmoid} & \multicolumn{1}{c|}{1.084613} & 0.059421 & \multicolumn{1}{c|}{63.99} & 97.74 \\ \hline
\textit{tanh} & \multicolumn{1}{c|}{0.897948} & 0.033934 & \multicolumn{1}{c|}{73.66} & 98.48 \\ \hline
\textit{RELU} & \multicolumn{1}{c|}{2.377840} & 0.004796 & \multicolumn{1}{c|}{75.33} & 99.54 \\ \hline
\textit{Leaky RELU (0.05)} & \multicolumn{1}{c|}{1.017293} & 0.006852 & \multicolumn{1}{c|}{79.33} & 99.31 \\ \hline
\textit{Leaky RELU (0.1)} & \multicolumn{1}{c|}{1.221875} & 0.005321 & \multicolumn{1}{c|}{77.00} & 99.52 \\ \hline
\textit{Leaky RELU (0.4)} & \multicolumn{1}{c|}{0.443687} & 0.008263 & \multicolumn{1}{c|}{79.00} & 99.24 \\ \hline
\textit{Leaky RELU (0.5)} & \multicolumn{1}{c|}{0.417411} & 0.018245 & \multicolumn{1}{c|}{79.33} & 98.58 \\ \hline
\textit{Leaky RELU (0.6)} & \multicolumn{1}{c|}{0.450320} & 0.014370 & \multicolumn{1}{c|}{77.33} & 98.76 \\ \hline
\textit{Leaky RELU (0.7)} & \multicolumn{1}{c|}{0.445662} & 0.021970 & \multicolumn{1}{c|}{74.66} & 98.40 \\ \hline
\end{tabular}
\caption{Validation loss and accuracy on multiple models with different activation functions.}
\label{tab:actfuncs}
\end{table}}


We can see that most of the activation functions performed better than 70 \% aside from the sigmoid function. This corresponds with the practical experience of many people since sigmoid is not used in neural networks anymore due to its problems. On the other hand, one of the best performing functions in our model is Leaky RELU. Surprising is that the tanh activation performed almost as better as RELU even though it's not very popular in the state-of-the-art models. 

\subsubsection{Loss function}

The goal of training is to adjust the weights of the network so that during the forward pass the network achieves the highest accuracy on data. The update of weights is performed during backpropagation and is based on the value of the loss function. The loss function penalizes the poor performance of our network classification and weight configuration. It has the following form: 
\begin{equation}
    L = \frac{1}{N} \sum_{i=1}^{N} L_{i} + \lambda R(W)
\end{equation}
where the first part of the expression is data loss, and the second part is regularization loss. Data loss is calculated as the average loss over all samples in the batch, where the batch size is denoted as $N$. Regularization loss is computed from a set of weights $W$ and parameter $\lambda$ is a hyperparameter that represents the contribution of regularization loss $R(W)$ to the loss function $L$ \cite{standford}.

In our work, we use cross-entropy loss to calculate the data loss component. It is used to calculate the loss of the prediction when the output is a probability value ranging from 0 to 1. Using softmax activation in the last fully-connected layer transforms our class scores into probability values. The cross-entropy loss works in a way that it increases as the predicted value deviates from the actual class label and decreases when the value is closer to the truth. In our case, we are classifying into multiple classes and cross-entropy is therefore calculated for each class separately using the following formula: 

\begin{equation}
    L_i = - \sum_{c=1}^{M} y_c log(p_c)
\end{equation}

where $M$ is number of classes, $y_c$ is binary indicator that expresses if the class $c$ is correct classification for example $i$ and $p_c$ is predicted probability value that example $i$ is of class $c$ \cite{standford}.

However, even if we find the configuration of weights that can classify each example, various sets of weights can achieve the same result. To eliminate the ambiguity we want the network to have a preference for certain weights. This is achieved through regularization loss which penalizes certain types of weights. 

One of the most common types of regularization losses and the one we are using in our thesis is L2 regularization. It penalizes large weights as it is believed that smaller weights mean a simpler model. And keeping the model simple prevents overfitting on training data and generalizes better for unseen data. L2 regularization has the following form: 

\begin{equation}
    R(W) = \sum_{i}^N W_i ^ 2 
\end{equation}
where set of weight $W$ is defined as $W = w_1, w_2, ... w_N$ \cite{standford}.








\subsubsection{Optimizer}

We mentioned that the loss function is used to measure the performance of our model to adjust the weights accordingly. The tool that provides this update is an optimizer, which uses the output of the loss function to determine how to update the weights. More specifically it uses the gradient of the loss function to predict the direction of the steepest slope to find local minima of the loss function. How much we move in that direction is determined by the parameter learning rate, which is essential in optimization and very hard to determine. 
There are several types of optimizers and each uses a different method to optimize the weights. In our work we are implementing the following optimizers:
\begin{itemize}
    \item Stochastic Gradient Descend (SGD) with momentum 
    \item Root Mean Square Prop (RMSProp)
    \item Adaptive Moment Estimation (ADAM) 
\end{itemize}

SGD with momentum is one of the simpler optimizers. It uses the negative direction of the gradient as well as an accumulated gradient over past iterations, which imitates velocity. It has the following form: 

\begin{equation}
\begin{split}
    v_t = \beta \cdot v_{t-1} + g_t \\
    W_t = W_{t-1} + LR \cdot v_t
\end{split}
\end{equation}

where $v_t$ is weighted average of previous gradients, $\beta$ is hyper paramater called momentum which controls how much the previous gradients $v_{t-1}$ contribute to current $v_t$, $g_t$ is gradient, $W$ is weight vector and $LR$ is hyper parameter learning rate \cite{pytorchoptim}.

Another popular approach in optimizers is to use second order derivative of loss function instead of just using gradient. One of them is RMSProp which is defined as:

\begin{equation}
\begin{split}
    v_t = \alpha \cdot v_{t-1} + (1 - \alpha) \cdot g_t^2 \\
    W_t = W_{t-1} - LR \cdot g_t / (\sqrt{v_t} + \epsilon) 
\end{split}
\end{equation}

where $v$ is a weighted average of previous squares of gradients, $\alpha$ is a hyperparameter that controls how $v_t$ contributes to $v_{t+1}$, and $\epsilon$ is a small constant value that prevents division by zero. Similar to SGD with momentum, $g$ is the gradient, $W$ is the weight vector and $LR$ is the learning rate \cite{pytorchoptim}.

One of the most popular optimizers and one that is the most complex is ADAM. It utilizes ideas from both RMSProp and SGD with momentum. The update of weights looks as follows: 

\begin{equation}
\begin{split}
    m_t = \beta_1 \cdot m_{t-1} + (1 - \beta_1) \cdot g_t \\
    v_t = \beta_2 \cdot v_{t-1} + (1 - \beta_2) \cdot g_t^2 \\
    \hat{m_t} = m_t / (1 - \beta_1^t) \\
    \hat{v_t} = v_t / (1 - \beta_2^t) \\
    W_t = W_{t-1} - LR \cdot \hat{m_t} / (\sqrt{\hat{v_t}} + \epsilon) 
\end{split}
\end{equation}

where $m_t$ is weighted average of previous gradients, $v_t$ is weighted average of previous squares of gradients, $\beta_1$ and $\beta_2$ are hyper parameters and $\hat{m_t}$, $\hat{v_t}$ are corrected values to prevent bias towards zero. Other parameters are same as in the previous formulas \cite{pytorchoptim}.

\subsubsection{Scheduler}

Neural networks require a large number of hyperparameters to tune. Some of them don't affect the performance to such an extent while others are essential for the model to be set properly. One of such essential ones is the learning rate of optimizers. If it's too low the training process is excessively long. On the other hand, if we set it to be too large, the optimization diverges from the minima. It may also not be optimal for the learning rate to stay the same during the whole training process. When we are near the local minima, large steps could cause us to stray away from it, while in the beginning, it could help us reach the minima faster.
One of the ways to improve the optimization process is to use learning rate schedulers. There are several different types of schedulers. Some reduce the learning rate after each epoch, others reduce it after they have reached a certain epoch or after a given set of predefined epochs. The reduction may be done by multiplying the learning rate by some factor or the user defines the specific set of learning rates that will be used. One of the common ways is to use the validation loss to determine when to reduce the learning rate.

\subsubsection{Early stopping algorithm}
When the network is trained for a large number of epochs, it leads to overfitting of the training data. To eliminate this problem we are using an early stopping algorithm. During training, the algorithm watches how the validation loss evolves. If it starts to degrade and doesn't improve for a certain amount of time (patience) the training stops. The algorithm also remembers the model with the best performance and uses this as the final model.

\subsubsection{Dropout}
Another way to prevent the network from overfitting is to use dropout. It is a technique that randomly removes some neurons from the computation of the network. Each neuron is dropped randomly with the probability of p, which is a hyperparameter defined by the user. This allows the network to not rely on specific neurons, prevents layers from co-adaptation, and makes the model essentially more robust. The dropout can be applied to fully-connected layers as well as convolutional layers, but not to the output layer as this would remove prediction for a certain class \cite{standford}.

\subsubsection{Data augmentation }
One of the ways to improve the performance of the network is to have more data. In the real world, it is usually hard to obtain more labeled data for the training of the network. This problem is solved by data augmentation which essentially just uses the data we already have but modifies them in some way. With image data, augmentation usually involves rotation, flipping, scaling, cropping, color jitter, or adding noise. 
In our case data augmentation is not needed since we are generating synthetic images and have an unlimited amount of data. However, after training the network on synthetic data we are using real images to fine-tune the model. All the real images used in this thesis were hand-cropped by us and it took a significant amount of time. To increase the amount of data and in much less time, we are using some augmentation techniques. However, our data are prone to damage using some of the mentioned techniques. If we rotate the image, we may lose the astronomical object that was on the edge of the image. The same goes for cropping or scaling. Color jittering and noise addition are not suitable options as well as this would compromise the strictly defined profile of astronomical objects. For this reason, we are using only the following techniques: 

\begin{itemize}
    \item rotation by 90, 180, 270 degrees
    \item flipping vertically and horizontally
\end{itemize}








%%%%%%%%%%%%%%%%%%%%%%% alternatives %%%%%%%%%%%%%%
%However to find the minima, we need to move in the negative direction of the gradient.  While the gradient represents the steepest slope 

%The gradient of the loss function is computed over the whole batch of images. Using the optimizer, 

% there are several types of optimizer, and each have different parameters. 

%Learning rate determines the speed of the optimalization. 
%Optimizer uses the gradient of the loss function to determine the direction of steepest decrease to minimize the loss.

%the goal of the optimizer is to find a set of weights that minimizes the loss function. The gradient of the loss function is calculated to determine the direction of the steepest increase. 

% The network usually has a predefined number of epochs,



\section{Data generator} \label{sec:sdgenerator}

For the training purposes of the network, we are developing a data generator. More specifically we are using the starGen and extending it to fit the needs of our thesis. The generator script is depicted in the Figure \ref{img:starGenActiviyDiagram}. 
In the beginning, the script reads the configuration file, which contains the general settings as well as settings for each generated object. The script generates multiple series and each series contains an arbitrary number of frames. The script supports the generation of stars, moving objects, clusters, and galaxies. Stars and galaxies are static in each frame, while clusters and moving objects move in consecutive frames. To make the images as realistic as possible, the script adds Gaussian and Poisson noise. It also supports defects such as hot pixels and cosmic rays. Other defects and noises are supported in a form of adding real BIAS, DARK and FLAT FIELD frames to the generated image. Other features of starGen include saving positions to TSV files, saving images in a form of FITS and PNG files, and plotting the images in the environment. It also includes the option to read the positions and brightness values of objects from the TSV file. 

\begin{figure}[h]
    \centering
    \includegraphics[width=\textwidth]{images/starGen2.jpg}
    \caption{Activity diagram of starGen script.}
    \label{img:starGenActiviyDiagram}
\end{figure}


\subsection{General settings}
General settings include image settings and also parameters of the generation itself. 
The script allows the user to set the number of generated series, the number of frames in the series, and the resolution of the generated image. Lots of parameters are binary indicators, that allow the user to choose whether he wants to save the images, plot the images and save the positions of objects. The user also defines the destination, where files are saved.  In case the positions of objects are read from the TSV file, the user is required to define the path to the file. For this approach to work, the number of frames in one series is not adjustable by the user but is determined by the file. 


\section{Objects and defects}
StarGen script supports multiple astronomical objects such as stars, galaxies, clusters, and moving objects. Apart from galaxies, all mentioned objects have two types of appearances: point and streak. In the script, we also generate defects such as hot pixels and three types of cosmic rays. Here we will explain how each object, in a morphological sense, is generated and show some examples. 

\subsection{Point}
Generation of point is performed using two dimensional Gaussian function with the following formula: 
\begin{equation} \label{eq:gauss}
    f(x,y) = \frac{1}{2 \pi \sigma^2} \cdot \exp{\left( - \frac{1}{2} \left( \frac{(x - x_0)^2 + (y - y0)^2 }{2 \sigma^2} \right) \right)}
\end{equation}
where $x_0$, $y_0$ are coordinates of the point and $\sigma$ is the standard deviation computed from the fwhm of the Gaussian. 

The point is generated by the following process. A blank image is created and the Gaussian function is used to generate a point on the image using the position and fwhm of the object. The values of the image are then normalized to interval <0,1> and multiplied by the brightness of the object. Poisson noise is applied and image values are clipped to the interval <0,65535>. This image of the object is then added to the input image. If we are generating only one object per frame, then the input image is blank and we could just return the image of the object. However, if we are generating frames with multiple objects then we need to add the new point to the input image. 

To compare the performance of our generator to real data we show the 3D profile of the real point and our generated point in a side-by-side view in the Figure \ref{fig:point3D}

\begin{figure}[H]
\centering
    \begin{subfigure}[t]{.4\textwidth}
        \centering
        \includegraphics[width=\textwidth]{images/realPoint3D.JPG}
        \caption{3D profile of the real point source.}
        \label{fig:point3D1}
    \end{subfigure}
    \begin{subfigure}[t]{.4\textwidth}
        \centering
        \includegraphics[width=\textwidth]{images/synpoint3D.JPG}
        \caption{3D profile of the synthetic point source.}
        \label{fig:point3D2}
    \end{subfigure}

    \caption{Comparison of 3D profiles between real and synthetic point source.}
    \label{fig:point3D}
\end{figure}

With point source, there aren't many parameters other than $fwhm$ and $brightness$ that can be configured. In the Figure \ref{fig:pointfwhm} we show generated images with various values of fwhm of the Gaussian function. 


\begin{figure}[!h]
\centering
    \begin{subfigure}{.23\textwidth}
        \centering
        \includegraphics[width=\textwidth]{images/fwhm2.png}
        \caption{fwhm = 2.}
        \label{fig:pointfwhm2}
    \end{subfigure}
    \begin{subfigure}{.23\textwidth}
        \centering
        \includegraphics[width=\textwidth]{images/fwhm3.png}
        \caption{fwhm = 3.}
        \label{fig:pointfwhm3}
    \end{subfigure}
    \begin{subfigure}{.23\textwidth}
        \centering
        \includegraphics[width=\textwidth]{images/fwhm4.png}
        \caption{fwhm = 4.}
        \label{fig:pointfwhm4}
    \end{subfigure}
    \begin{subfigure}{.23\textwidth}
        \centering
        \includegraphics[width=\textwidth]{images/fwhm5.png}
        \caption{fwhm = 5.}
        \label{fig:pointfwhm5}
    \end{subfigure}

    \caption{Generated images of point source with different values of fwhm.}
    \label{fig:pointfwhm}
\end{figure}


\subsection{Streak}
The streak source is generated using multiple overlapping two-dimensional Gaussian functions with the same Formula \ref{eq:gauss}. The process of the generation is similar to the point source but other parameters other than $\sigma$ and $brightness$ need to be taken into an account. The rotation of the streak relative to the positive x-axis is determined by parameter $alpha$. And half-length of the streak is set by $length$. Figure \ref{fig:streak3D} shows the comparison of the real and synthetic 3D profile of the streak source. 


\begin{figure}[!h]
\centering
    \begin{subfigure}[t]{.4\textwidth}
        \centering
        \includegraphics[width=\textwidth]{images/realStreak3D.JPG}
        \caption{3D profile of the real streak source.}
        \label{fig:streak3D1}
    \end{subfigure}
    \begin{subfigure}[t]{.4\textwidth}
        \centering
        \includegraphics[width=\textwidth]{images/synStreak3D.JPG}
        \caption{3D profile of the synthetic streak source.}
        \label{fig:streak3D2}
    \end{subfigure}

    \caption{Comparison of 3D profiles between real and synthetic streak source.}
    \label{fig:streak3D}
\end{figure}

In the Figure \ref{fig:pointfwhm} we have shown how fwhm affects the size of the point source. With streak source, the fwhm affects its width. As the name suggests, the length determines how long the streak is and alpha describes its rotation. Examples of generated streak sources with different settings of parameters are shown in the Figure \ref{fig:streakspar}. 

\begin{figure}[!h]
\centering
    \begin{subfigure}{.23\textwidth}
        \centering
        \includegraphics[width=\textwidth]{images/streakA.png}
        \caption{fwhm = 2, alpha = 0, length = 3.}
        \label{fig:streakA}
    \end{subfigure}
    \begin{subfigure}{.23\textwidth}
        \centering
        \includegraphics[width=\textwidth]{images/streakB.png}
        \caption{fwhm = 3, alpha = 90, length = 5.}
        \label{fig:streakB}
    \end{subfigure}
    \begin{subfigure}{.23\textwidth}
        \centering
        \includegraphics[width=\textwidth]{images/streakC.png}
        \caption{fwhm = 4, alpha = 135, length = 7.}
        \label{fig:streakC}
    \end{subfigure}
    \begin{subfigure}{.23\textwidth}
        \centering
        \includegraphics[width=\textwidth]{images/streakD.png}
        \caption{fwhm = 5, alpha = 45, length = 9.}
        \label{fig:streakD}
    \end{subfigure}

    \caption{Generated images of streak source with different values of fwhm, alpha, and length.}
    \label{fig:streakspar}
\end{figure}


\subsection{Galaxy}
There are various types of galaxies, but in our work, we are only generating elliptical galaxies. If we look at the 3D profile of a real galaxy (Figure \ref{fig:galaxy3D1}) we can see that it resembles the Cauchy function. However, the Cauchy function would not be suitable since we can only control the scale parameter $\gamma$ in both directions simultaneously. In our case we want different $\gamma$ in the x-direction and different $\gamma$ in the y-direction to create an ellipse-like appearance. For this purpose we use a bivariate Gaussian function defined by the following formula: 

\begin{equation} \label{eq:bigaus}
    f(x,y) = \exp \left(  - \left( a \cdot (x-x_0)^2 + 2b \cdot (x - x_0) \cdot (y - y_0) + c \cdot (y - y0)^2 \right)\right)
\end{equation}

where $x_0$, $y_0$ define the position of generated galaxy and $a$, $b$, $c$ are parameters defined as:

\begin{equation}
    \begin{split}
        a = \frac{\cos^2{\theta}}{2\sigma_x^2} + \frac{\sin^2{\theta}}{2\sigma_y^2}, \\
    b =  \frac{\sin{2\theta}}{4\sigma_x^2} - \frac{\sin{2\theta}}{4\sigma_y^2}, \\
    c = \frac{\sin^2{\theta}}{2\sigma_x^2} + \frac{\cos^2{\theta}}{2\sigma_y^2}
    \end{split}
\end{equation}

where $\sigma_x$ and $\sigma_y$ are standard deviations of the Gaussian function in direction x, y and parameter $\theta$ defines the anticlockwise rotation of the generated galaxy. 

To match the profile of the synthetic galaxy to the real one, we use a linear combination of two bivariate Gaussian functions to simulate the steep decrease from the peak similarly to the Cauchy function. The Figure \ref{img:gaussTwo} illustrates what we are aiming for in one dimension. The green Gaussian simulates the large diffuse outer area of the galaxy and the orange Gaussian mimics the sharp concentrated core with high brightness. Adding the green and orange Gaussian creates the blue one, which is our desired profile of the galaxy.

\begin{figure}[h]
    \centering
    \includegraphics[width=.7\textwidth]{images/gaussians.JPG}
    \caption{Combination of two Gaussian functions.}
    \label{img:gaussTwo}
\end{figure}

In our system, we first generate each Gaussian separately using the Equation \ref{eq:bigaus}. The standard deviations of the diffuse Gaussian are set by the user in the configuration using $sigmaX$ and $sigmaY$. For the sharp Gaussian user defines parameter $sigmaFactor$ and standard deviations are calculated using Equation \ref{eq:sigmaGalaxy} mentioned in the previous chapter. The position ($x_0$,$y_0$) is the same for both Gaussians as well the rotation $\theta$. After both Gaussians are generated, the pixel values of both images are normalized into the interval <0,1>. 

In the configuration, the user defines the $brightness$ of the galaxy, more specifically the intensity of the inner core. The user also specifies the brightness of the diffuse Gaussian in a form of using $brightnessFactor$, and the brightness is calculated using Equation \ref{eq:brightnessGalaxy} from the previous chapter. 
To account for two Gaussians being added together, the sharp Gaussian has the intensity of $(1 - brightnessFactor) \cdot brightness$ and the diffuse Gaussian has $brightnessFactor \cdot brightness$ intensity. After adding the functions together, the intensity of the core is $brightness$ as defined in the configuration. 

The final galaxy profile is generated using following linear combination of Gaussians:
\begin{equation}
    galaxy = A \cdot sharpGauss + B \cdot diffuseGauss
\end{equation}
where $A$ is $(1 - brightnessFactor) \cdot brightness$ and $B$ is $brightnessFactor \cdot brightness$. 

At the end, when the galaxy is generated, Poisson noise is applied, pixel values are clipped to the interval <0,65535> and the image is added to the input image the same way as with the point and streak source. 

Again, to compare generated galaxy using the following process to a real elliptical galaxy we have created a side-by-side view of their 3D profiles in the Figure \ref{fig:galaxy3D}.

\begin{figure}[!h]
\centering
    \begin{subfigure}[t]{.4\textwidth}
        \centering
        \includegraphics[width=\textwidth]{images/realGalaxy3D.JPG}
        \caption{3D profile of the real galaxy.}
        \label{fig:galaxy3D1}
    \end{subfigure}
    \begin{subfigure}[t]{.4\textwidth}
        \centering
        \includegraphics[width=\textwidth]{images/syngalaxy3D2.JPG}
        \caption{3D profile of the synthetic galaxy.}
        \label{fig:galaxy3D2}
    \end{subfigure}

    \caption{Comparison of 3D profiles between real and synthetic galaxies.}
    \label{fig:galaxy3D}
\end{figure}

With galaxy we have wider range of parameters to control. With $sigmaFactor$ it proved to be the best to use values ranging from 0.2 to 0.4 and with $brightnessFactor$ values from 0.6 to 0.8. In the Figure \ref{fig:galaxypar} we show how other parameters like $sigmaX$, $sigmaY$ and $alpha$ affect the generated galaxy. All images have $sigmaFactor$ set as 0.3, and $brightnessFactor$ to 0.6.  


\begin{figure}[!h]
\centering
    \begin{subfigure}[t]{.23\textwidth}
        \centering
        \includegraphics[width=\textwidth]{images/galaxyA.png}
        \caption{sigmaX = 3, sigmaY = 3, alpha = 0.}
        \label{fig:galaxyA}
    \end{subfigure}
    \begin{subfigure}[t]{.23\textwidth}
        \centering
        \includegraphics[width=\textwidth]{images/galaxyB.png}
        \caption{sigmaX = 3.5, sigmaY = 6.5, alpha = 135.}
        \label{fig:galaxyB}
    \end{subfigure}
    \begin{subfigure}[t]{.23\textwidth}
        \centering
        \includegraphics[width=\textwidth]{images/galaxyC.png}
        \caption{sigmaX = 5.5, sigmaY = 2.5, alpha = 45.}
        \label{fig:galaxyC}
    \end{subfigure}
    \begin{subfigure}[t]{.23\textwidth}
        \centering
        \includegraphics[width=\textwidth]{images/galaxyD.png}
        \caption{sigmaX = 5, sigmaY = 5, alpha = 0.}
        \label{fig:galaxyD}
    \end{subfigure}

    \caption{Generated images of galaxy with different values of sigmaX, sigmaY and alpha.}
    \label{fig:galaxypar}
\end{figure}

\subsection{Hot pixels}
Hot pixels are just individual pixels with very high brightness and they are generated as such. The user defines their brightness and count in the configuration. Another added parameter is that he can specify the value of the random seed which will be used when generating their positions since hot pixels are in the same position in each frame in the series. This way the system can generate multiple series with the same set of hot pixels. For the purpose of generating data for the network, we added an option to generate random positions for each series since we want to have the pixel at different locations of the image. Examples of generated hot pixels at different locations are shown in the Figure \ref{fig:hotpixelspar}.

\begin{figure}[!h]
\centering
    \begin{subfigure}[t]{.23\textwidth}
        \centering
        \includegraphics[width=\textwidth]{images/hotpixelA.png}
    \end{subfigure}
    \begin{subfigure}[t]{.23\textwidth}
        \centering
        \includegraphics[width=\textwidth]{images/hotpixelB.png}
    \end{subfigure}
    \begin{subfigure}[t]{.23\textwidth}
        \centering
        \includegraphics[width=\textwidth]{images/hotpixelC.png}
    \end{subfigure}
    \begin{subfigure}[t]{.23\textwidth}
        \centering
        \includegraphics[width=\textwidth]{images/hotpixelD.png}
    \end{subfigure}

    \caption{Generated images of hot pixels at different positions in the image. }
    \label{fig:hotpixelspar}
\end{figure}

\subsection{Cosmic rays}
Cosmic rays can be grouped into three categories based on their appearance on the image: spots, worms, and tracks. Each type is generated separately with a different algorithm. 

Let's first consider spots. As the name suggests they resemble small spots on the image, that have very high brightness compared to the background. They are generated using an iterative method. First, the random position on the image is selected and the brightness value is assigned to it. Then another pixel is chosen from the four-neighborhood of the previous pixel and assigned a different brightness value. This continues until we create a spot with an exact number of pixels as was defined in the configuration. If the chosen neighboring pixel has already brightness assigned to it then another pixel is chosen. In the Figure \ref{fig:spotsPar} we show some examples of generated spots. 
\begin{figure}[!h]
\centering
    \begin{subfigure}[t]{.23\textwidth}
        \centering
        \includegraphics[width=\textwidth]{images/spotA.png}
    \end{subfigure}
    \begin{subfigure}[t]{.23\textwidth}
        \centering
        \includegraphics[width=\textwidth]{images/spotB.png}
    \end{subfigure}
    \begin{subfigure}[t]{.23\textwidth}
        \centering
        \includegraphics[width=\textwidth]{images/spotC.png}
    \end{subfigure}
    \begin{subfigure}[t]{.23\textwidth}
        \centering
        \includegraphics[width=\textwidth]{images/spotD.png}
    \end{subfigure}

    \caption{Generated images of cosmic rays - spots. }
    \label{fig:spotsPar}
\end{figure}

On the other hand, worms have a structure of a curved polyline. They are also generated using an iterative algorithm. First, we select a random position on the image and direction from 8 directions shown in the Figure \ref{img:8directions}. We move in this direction for a few pixels and fill each pixel on the way. Then another direction is selected, but it needs to differ from the previous direction only by one. So for example, if the previous direction was 8 then the next direction can be 7 or 1. This continues until we create a worm with the number of pixels defined from the configuration. Examples of some generated worms are depicted in the Figure \ref{fig:wormspar}. 

\begin{figure}[h]
    \centering
    \includegraphics[width=.23\textwidth]{images/8directions.png}
    \caption{Diagram of 8 directions.}
    \label{img:8directions}
\end{figure}

\begin{figure}[!h]
\centering
    \begin{subfigure}[t]{.23\textwidth}
        \centering
        \includegraphics[width=\textwidth]{images/wormA.png}
    \end{subfigure}
    \begin{subfigure}[t]{.23\textwidth}
        \centering
        \includegraphics[width=\textwidth]{images/wormB.png}
    \end{subfigure}
    \begin{subfigure}[t]{.23\textwidth}
        \centering
        \includegraphics[width=\textwidth]{images/wormC.png}
    \end{subfigure}
    \begin{subfigure}[t]{.23\textwidth}
        \centering
        \includegraphics[width=\textwidth]{images/wormD.png}
    \end{subfigure}

    \caption{Generated images of cosmic rays - worms. }
    \label{fig:wormspar}
\end{figure}

Tracks resemble lines on the image, but generating just diagonal pixels to create them is not ideal, since tracks in real observations don't follow the precise shape of lines. The process of generating them is very similar to generating worms, with just one modification. Instead of choosing another direction every time, we establish two directions at the beginning and only follow these two. Again these two directions need to differ only by one number from the diagram of directions (Figure \ref{img:8directions}). This creates pseudo-lines as shown in the Figure \ref{fig:trackspar} where we generated some examples of tracks. 

\begin{figure}[!h]
\centering
    \begin{subfigure}[t]{.23\textwidth}
        \centering
        \includegraphics[width=\textwidth]{images/trackA.png}
    \end{subfigure}
    \begin{subfigure}[t]{.23\textwidth}
        \centering
        \includegraphics[width=\textwidth]{images/trackB.png}
    \end{subfigure}
    \begin{subfigure}[t]{.23\textwidth}
        \centering
        \includegraphics[width=\textwidth]{images/trackC.png}
    \end{subfigure}
    \begin{subfigure}[t]{.23\textwidth}
        \centering
        \includegraphics[width=\textwidth]{images/trackD.png}
    \end{subfigure}

    \caption{Generated images of cosmic rays - tracks. }
    \label{fig:trackspar}
\end{figure}

\section{Defects and noises} \label{sec:defects}

There are various ways the CCD image can be distorted or corrupted and also numerous sources of these defects. 


\subsection{Statistical noise}

    %%%%%%%%%%%%%%%%%%%%%%%%%%%% 
    \subsubsection{Photon noise}
    
    The distribution of photons falling on the CCD chip obeys simple counting statistics. 
    Let's assume the CCD chip is illuminated with constant uniform light and that the sensitivity of each pixel is the same. 
    For each finite moment, there is an amount of photons detected on the chip which varies.
    If we were to plot the amount of photons detected and the occurrences of each amount in a histogram, we would quickly realise that the distribution is following a Poisson distribution. As we are dealing with large amount of photons, Poisson distribution can be approximated with the Gaussian function: 
    
    \begin{equation}
     \label{eqn:gaussian}
        g(x) = \frac{1}{\sigma \cdot \sqrt{2 \pi}} \cdot exp \left(  - \frac{1}{2} \cdot \frac{(x - \mu)^2}{\sigma^2} \right)   
    \end{equation}
    
    where $\mu$ is the average or the most common amount of photons detected and $\sigma$ is the standard deviation. 
    The number of photons detected on the chip can also be referred to as a signal $S$ \cite{romanishin2006introduction}.
    
    Let's denote, that $S_{star}$ is the signal from the star and $S_{sky}$ is the signal from the sky. From \cite{romanishin2006introduction} the noise from the star $\sigma_{star}$ and the noise from the sky $\sigma_{sky}$ can be expressed as:
    
    \begin{equation}
    \label{eqn:starskynoise}
        \sigma_{star} = \sqrt{S_{star}} \quad \sigma_{sky} = \sqrt{S_{sky}}
    \end{equation} 

    and addition of these noises is done in a following manner:
    \begin{equation}
    \label{eqn:additionNoises}
        \sigma = \sqrt{\sigma_{star}^2 + \sigma_{sky}^2}
    \end{equation}

\subsection{Internal}

Internal defects and noises are caused by issues within the CCD chip. This includes mechanical and electrical issues. 
 

    %%%%%%%%%%%%%%%%%%%%%%%%%%%% 
    \subsubsection{Bias}
    
    In the process of accumulating photo-electrons in the CCD chip, let's suppose that each pixel accumulated 2 photo-electrons on average. Given how the analog to digital conversion works, a number of counts $N_{counts}$ is proportional to a number of photo-electrons $N_{elec}$ accumulated in the pixel and is expressed by the following: 
        
    \begin{equation}
    \label{eqn:bias}
        N_{elec}  = gain \cdot N_{counts}
    \end{equation}
    
    For the simplicity of the example let's suppose that the gain is 1 e-/ADU. According to the formula, each pixel should have 2 counts, but due to the readout noise, the number of counts will follow the Gaussian distribution. The mean of the Gaussian is 2 counts, with a standard deviation given by the readout noise, which could be as much as 3 counts. This would mean that the number of counts in some pixels would be negative. However, the ADC cannot represent negative numbers, which means that the data would be corrupted \cite{articleCCDartifacts} \cite{phy217}.
    To fix this issue, bias voltage was applied to the CCD detector. It is a constant offset voltage, which means that even if the pixel contained no photo-electrons, the ADC returns a high positive value. This solves the issue with negative values. The constant bias voltage is identified through the BIAS frame \cite{articleCCDartifacts} \cite{phy217}.
    
        
 
    %%%%%%%%%%%%%%%%%%%%%%%%%%%% 
    \subsubsection{Readout noise}
    
    Reading out the image from the CCD chip includes charge shifting through the chip followed by conversion to the voltage. Due to the limitations in the current technology, the voltage in the pixel cannot be measured perfectly and readout noise is introduced. 
    The noise is expressed as the number of electrons per pixel. This number is characteristic for each CCD chip and is usually defined by manufacturers.
    Readout noise is generated during the process of reading the image, therefore it is independent of the exposure time of the image. However, the amount of noise increases with the higher speed of the readout \cite{articleCcdOnline}.
    
    From \cite{bolte15} readout noise in the aperture can be expressed as: 
    
    \begin{equation}
    \label{eqn:readoutNoise}
         \sigma_{RN} = \sqrt{N_{pix} \cdot {RN}^2 }
    \end{equation}
    
    where $N_{pix}$ is a number of pixels in the aperture and $RN$ is readout noise in electrons per pixel.

    Note, that the readout noise is not following Poisson distribution and is not related to the signal coming from stars or sky \cite{matfyzpress01}. Therefore Formula \ref{eqn:starskynoise} cannot be used, when expressing the readout noise. 

    %%%%%%%%%%%%%%%%%%%%%%%%%%%% 
    \subsubsection{Dark current}
    
    Photo-electrons in CCD pixels are produced by photons from incoming light. However, thermal excitation can also produce electrons in pixels, and it's impossible to differentiate between them and photo-electrons coming from the light \cite{phy217}.
    This effect is called dark current and is expressed in electrons per second per pixel at the defined temperature. The amount of dark current is linearly proportional to chip temperature and exposure time. To minimize the effect, CCD chips are usually cooled down to a temperature below 0°C \cite{articleCcdOnline}.
    
    The amount of dark current is also affected by pixel size, chip architecture, and production technology. Therefore manufacturers usually define the amount of dark current for specific models of CCD chips.
    Even after the chip is cooled, the amount of dark current is not negligible. To detect the dark current in the image, DARK frames were created \cite{articleCcdOnline}. The noise from the dark current is also following Poisson distribution \cite{matfyzpress01}. 
    
    From \cite{bolte15} the noise caused by dark current in our aperture can be expressed as:
    
    \begin{equation}
    \label{eqn:darkcurrent}
        \sigma_{DC} = \sqrt{D \cdot N_{pix} \cdot t}
    \end{equation}

    
    where $D$ is dark current in $e^-/second/pixel$, $N_{pix}$ is number of pixels in the aperture and $t$ is the exposure time.
    
    
    \begin{figure}[!h]
    \centering
        \begin{subfigure}{.3\textwidth}
        \centering
            \includegraphics[width=\textwidth]{images/hotpixels.png}
            \caption{Hot pixels.}
            \label{fig:hotpixels}
        \end{subfigure}
        \begin{subfigure}{.3\textwidth}
            \centering
            \includegraphics[width=\textwidth]{images/deadcolumns.jpg}
            \caption{Dead columns.}
            \label{fig:deadcolumns}
        \end{subfigure}
        
        \vspace*{4mm}
        
        \begin{subfigure}{.3\textwidth}
            \centering
            \includegraphics[width=\textwidth]{images/traps2.jpg}
            \caption{Traps.}
            \label{fig:trap}
        \end{subfigure}
        \begin{subfigure}{.3\textwidth}
            \centering
            \includegraphics[width=\textwidth]{images/saturationtrail.png}
            \caption{Saturation trail.}
            \label{fig:saturationtrail}
        \end{subfigure}
        \caption{Examples of some internal defects present on FITS images acquired by AGO70.}
        \label{fig:internaldefects}
    \end{figure}
    
    
    
    
    
    %%%%%%%%%%%%%%%%%%%%%%%%%%%% 
    \subsubsection{Hot Pixels}
    
    Hot pixels (Figure \ref{fig:hotpixels}) are individual pixels that have a very high dark current, meaning they are really bright or fully saturated. This is related to the defect of the chip which was either created during the manufacturing or by aging. Fully saturated pixels have reached their limit of detected photo-electrons, which means that any incoming photons will not generate any electrons. 
    As this is the result of the CCD chip issue, their positions on the image are fixed. This makes them easy to detect on a DARK frame.
    % citacia

        
    %%%%%%%%%%%%%%%%%%%%%%%%%%%% 
    \subsubsection{Dead Pixels}
    
    Dead pixels are also individual pixels, which on the contrary are not sensitive to any light and are completely dark. Similar to hot pixels, their positions are fixed from image to image and can be detected on the DARK frame.
    % citacia
    
    %%%%%%%%%%%%%%%%%%%%%%%%%%%% 
    \subsubsection{Dead columns} 
    
    Dead columns (Figure \ref{fig:deadcolumns})are columns of pixels corrupted in some way. %as depicted in the figure \ref{fig:deadcolumns}. 
    They can be either completely dark (not sensitive to any incoming photons) or fully saturated (reached the limit of detected photo-electrons). Apart from that, dead columns can also have a constant value added to what their real value should be. 
    They are various reasons for this defect, mechanical or electronical \cite{articleCCDartifacts}.
    
    
    
    %%%%%%%%%%%%%%%%%%%%%%%%%%%% 
    \subsubsection{Traps} 
    
    Traps (Figure \ref{fig:trap}) occur when some pixels on the CCD chip are damaged in a way that they cannot transfer photo-electrons \cite{articleCCDartifacts}. 
    As mentioned earlier during the process of reading the image, the charge is shifted through pixels. All rows are shifted one row below and the last row is shifted to the horizontal register from which the charge is shifted to the ADC \cite{articleCcdOnline}. However, if a pixel is damaged, the charge cannot pass through during the shifting of the rows. 
    
    Let's assume, the damaged pixel is in the first column and on the 4th-row counting from the top. Note that damage is only present in the first column, which contained the damaged pixel and other columns are unaffected.
    
    Pixels below the 4th row are read normally. However, pixels in the first three rows cannot be shifted through the damaged 4th pixel. Electrons in these pixels are therefore trapped and their charge cannot be converted to counts. This will result in completely dark pixels in these first three rows. %An example of a vertical trap is shown in the figure \ref{fig:trap} 
    
    Note, that traps are vertical when the CCD chip is read horizontally, by shifting charge from one row to another. However, in case the chip is read vertically and the charge is shifting through columns, traps will be horizontal. 


    %%%%%%%%%%%%%%%%%%%%%%%%%%%%
    \subsubsection{Saturation trail}
    
    Each pixel on the CCD chip has a certain amount of photo-electrons it can store. Some very bright sources can illuminate pixels to the point where they are completely filled up. As the pixel is filled to its limit, photo-electrons from the saturated pixel can start leaking to adjacent pixels. These extra electrons create a trail of saturated pixels - saturation trail (Figure \ref{fig:saturationtrail}) \cite{articleCCDartifacts}.
    
    Similar to traps, trails are either horizontal or vertical and it depends on the way the CCD chip is read. 
    



\subsection{External}

External noises and defects are caused by external forces such as moonlight, sunlight, particles in the universe, the atmosphere, and many others. All these effects are discussed below.

    
    %%%%%%%%%%%%%%%%%%%%%%%%%%%% 
    \subsubsection{Sky background noise} 
    
    One of the most dominant sources of noise in CCD images is the noise from the sky. This is caused by various sources of light present in the sky as well as some photons flying through space. 
    %For stars to be seen, their signal must be brighter than the noise coming from the sky. However, this does not mean that fainter stars will never be seen. The amount of signal is not as important as the ratio of the signal to the current noise in the sky. As this is a common problem in astronomy, this ratio is well known and is called signal-to-noise ratio - SNR. The higher the ratio, the easier it is to see the signal from the stars.
    The sky noise is not constant and changes depending on various circumstances. As the sky gets brighter, the noise from the sky increases as well. This is caused by light pollution. Decreasing noise can be therefore observed when the sky is the darkest. Since the signal from stars has a constant value and the noise has a decreasing tendency this results in higher SNR of star signals \cite{romanishin2006introduction}. 
    
    Sky observations are usually done when the sky is darkest and at remote places to avoid light pollution. However even this does not guarantee the perfect dark sky. Various atomic processes are happening in the atmosphere. Molecules and atoms are constantly colliding with each other, which causes the excitation of atoms. This causes the air to glow which is yet another source of light contributing to the sky background noise. 
    In the case of space-based observations, the sky is darker compared to observations from Earth as there is no light coming from artificial sources and the atmosphere. However, another dominant source of light is present, which is called zodiacal light. This is sunlight scattered by dust particles and is present in the whole Solar system \cite{romanishin2006introduction}.
    
    %With low exposure time, the amount of sky noise is also very low and electronic noises are dominating the overall noise.
    
    %%%%%%%%%%%%%%%%%%%%%%%%%%%% 
    \subsubsection{Stray light} 

    Stray light (Figure \ref{fig:straylight})is caused by the reflection of light from optical surfaces like tubes or domes. Various sources of light can be reflected. This includes scattered moonlight which causes a bright gradient present in the image. Artificial sources of light such as diodes or internal lights can also be reflected on the image.
    % citacia
    
    %%%%%%%%%%%%%%%%%%%%%%%%%%%% 
    \subsubsection{Diffuse sources} 
    
    Diffuse sources such as galaxies (Figure \ref{fig:diffgalaxy}), nebulas, and comets (Figure \ref{fig:comet}) are also contributing to image corruption. They are not explicitly defects or noises, but their presence in the image can cause problems during photometry as they are contributing to the sky‘s brightness. 

        
    %%%%%%%%%%%%%%%%%%%%%%%%%%%%
    \subsubsection{Dust on filter}
    
    Dust particles (Figure \ref{fig:dust}) present on the lens or filters of the aperture can also cause problems in the image. Apart from other defects mentioned, which are usually only locally on the image, dust particles affect the image globally. The particles cast shadows on the detector and this manifests on the image in a form of dark rings \cite{articleCCDartifacts}.
    FLAT FIELD frame can detect these defects and thus are later corrected in a photometric reduction \cite{articleParimucha}.
    
    
    \begin{figure}[!h]
    \centering
        \begin{subfigure}{.3\textwidth}
        \centering
            \includegraphics[width=\textwidth]{images/straylight.png}
            \caption{Stray light.}
            \label{fig:straylight}
        \end{subfigure}
        \begin{subfigure}{.3\textwidth}
            \centering
            \includegraphics[width=\textwidth]{images/galaxyreal.jpg}
            \caption{Galaxy.}
            \label{fig:diffgalaxy}
        \end{subfigure}

        \vspace*{4mm}

        \begin{subfigure}{.3\textwidth}
            \centering
            \includegraphics[width=\textwidth]{images/comet.jpg}
            \caption{Comet.}
            \label{fig:comet}
        \end{subfigure}
        \begin{subfigure}{.3\textwidth}
            \centering
            \includegraphics[width=\textwidth]{images/dust.jpg}
            \caption{Dust ring.}
            \label{fig:dust}
        \end{subfigure}
        \caption{Examples of some external defects present on FITS images acquired by AGO70.}
        \label{fig:externaldefects}
    \end{figure}
 
    %%%%%%%%%%%%%%%%%%%%%%%%%%%% 
    \subsubsection{Cosmic rays}
    
    %A CCD chip is an instrument detecting photons emitting from light sources. Each photon creates a single electron in a specific pixel of the chip. 
    Flying through space are also high-energy particles called cosmic rays.  Each particle can excite hundreds or thousands of electrons, which can also cross through multiple pixels. 
    A cosmic ray hitting the CCD chip is an unpredictable event that can occur randomly affecting the image locally \cite{imageProc}. The temperature of the chip or its defects do not affect the event of cosmic rays hitting the chip. However, the longer the exposition time, the more cosmic rays hit the chip. 
    Cosmic rays can manifest as a single very bright pixel but sometimes it affects several adjacent pixels. The created object has strong and asymmetric features with sharp edges \cite{irafArticle}.
    
    Their profile can be classified into three categories \cite{inbookCosmics} which are shown in the Figure \ref{img:cosmicraysreal}  
        \begin{itemize}
            \item spot-like - resemble dots
            \item track-like - resemble straight lines
            \item worm-like - resemble polylines and curves
        \end{itemize}
    
    
    They could also be described as a group of connected pixels, which have count values higher than the background and at least one pixel having a significantly high value \cite{inbookCosmics}.
    
    When it comes to ground-based observations, they don't suffer from cosmic rays as much as space-based. Mainly because the amount of cosmic rays significantly increases in space. 

    \begin{figure}[!h]
    \centering
        
        \begin{subfigure}{\textwidth}
            \centering
            \includegraphics[width=.13\textwidth]{images/spot1.png}
            \includegraphics[width=.13\textwidth]{images/spot2.png}
            \includegraphics[width=.13\textwidth]{images/spot3.png}
            \includegraphics[width=.13\textwidth]{images/spot4.png}
            \caption{Spots.}
        \end{subfigure}
        \vskip\baselineskip
        \begin{subfigure}{\textwidth}
            \centering
            \includegraphics[width=.13\textwidth]{images/track1.png}
            \includegraphics[width=.13\textwidth]{images/track2.png}
            \includegraphics[width=.13\textwidth]{images/track3.png}
            \includegraphics[width=.13\textwidth]{images/track4.png}
            \caption{Tracks.}
        \end{subfigure}
        \vskip\baselineskip
        \begin{subfigure}{\textwidth}
            \centering
            \includegraphics[width=.13\textwidth]{images/worm1.png}
            \includegraphics[width=.13\textwidth]{images/worm2.png}
            \includegraphics[width=.13\textwidth]{images/worm3.png}
            \includegraphics[width=.13\textwidth]{images/worm4.png}
            \caption{Worms.}
        \end{subfigure}
        
        \caption{Examples of three categories of cosmic rays. Source \cite{cosmicrayimageall}.}
        \label{img:cosmicraysreal}  
    \end{figure}







%%% useful webpages
% https://ml-cheatsheet.readthedocs.io/en/latest/regularization.html#data-augmentation
% https://d2l.ai/chapter_computational-performance/index.html
