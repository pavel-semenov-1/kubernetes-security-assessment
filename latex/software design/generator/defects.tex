\subsection{Supported defects and noises}
To make images realistic, defects and noises that corrupt real astronomical images were added to the generator. 
This includes noises such as Gaussian noise and Poisson noise and defects like hot pixels and cosmic rays. We are aware that some noises and defects are missing. However, the data generator was not the primary focus of the thesis and was only a secondary tool to generate images for training purposes. Yet to make up for missing noises, we added the option to use real BIAS, DARK and FLAT FIELD frames in the generation. These real frames already include readout noise, bias voltage, dark current, dead columns, dust rings, vignette, and others (more info in Section \ref{sec:defects}). 

\subsubsection{Gaussian noise}
Gaussian noise is used to simulate the sky background noise. It is applied to each frame separately to keep the randomness in the images. The user can define the $mean$ and standard deviation ($std$) of the noise.

\subsubsection{Poisson noise}
Poisson noise is applied to each generated object to simulate the photons falling onto the chip. In the configuration file, the user can define if he wants to apply the Poisson noise using the $applyPoisson$ boolean parameter. 

\subsubsection{Hot pixels}
In the generation of hot pixels on the image, the user can set their $count$ and $brightness$. In generated series, hot pixels are static and stay at the same position in all frames. 

\subsubsection{Cosmic rays}
In the configuration file, the user can set the number of generated cosmic rays with $count$, as well as their $brightness$. In Section \ref{sec:defects} we mentioned three different types of cosmic rays: spots, tracks, and worms. Spots usually have fewer pixels than tracks and worms, which is why the user can define the number of pixels using $spotPixelCount$ for spots and $pixelCount$ for tracks and worms. Cosmic rays are generated randomly for each frame since in the real observations they occur randomly as well.

\subsubsection{BIAS, DARK, FLAT FIELD frames}
When using real frames, the user must define the path to the real images using parameter $dataDir$. Important to note, that the images must have the same dimensions as the generated image, or else it will not work correctly. The same real frames are applied to each frame after all objects are generated. First, we multiply the image with the values from the FLAT FIELD frame, which defines the pixel sensitivity. DARK and BIAS frames are added afterward. DARK frames usually already contain bias voltage, so we don't need to apply the BIAS frame.



