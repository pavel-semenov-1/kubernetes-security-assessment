\subsection{Supported objects}
As already mentioned, the generator supports multiple astronomical objects like stars, galaxies, moving objects, and clusters of moving objects. In this section, we will explain the parameters of each object. 

Note that all number parameters defined by the user are defined in the form of a range, where the user defines the minimum and maximum possible value and the script chooses a specific value from this interval. 

\subsubsection{Stars}
The generator allows user to define the number of generated stars using $count$, their maximal $brightness$, and $fwhm$ of their profile. Stars can be generated either as a point source, where the PSF is Gaussian or as a streak source, which consists of multiple overlapping Gaussians. This is determined by parameter $method$. In case the user chooses the stars to appear as streaks, additional parameters such as their rotation $alpha$ and $length$ need to be set. The rotation $alpha$ is anticlockwise and the values range from 0 to 360 degrees. The length of the streak is measured as half-length and the unit is $\sigma$ of the Gauss function. During one series, stars are static and they stay in the same position. All generated stars have the same $fwhm$, $alpha$ and $length$, while the $brightness$ differs. 

\subsubsection{Moving objects}
Similar to stars user can define the number of moving objects with parameter $count$, their $brightness$, $fwhm$, $method$ of appearance, with the same additional parameters $alpha$ and $length$. However, moving objects are not static and they change positions in consecutive frames. To adjust how much they move in frames, the additional parameter $speed$ was defined. The value of the $speed$ parameter is described as the percentage of the image traveled by the object in one series and it is used in the following manner: 

\begin{equation}
    \Delta = \frac{dim \cdot speed}{frames} 
\end{equation}

where $\Delta$ defines the traveled distance in pixels between two consecutive frames, $dim$ is the smaller dimension of the image, and $frames$ is the number of frames in one series. 

The direction in which the object moves is controlled by the rotation $alpha$ even if the moving object is a point source. The script supports generation of multiple moving objects and each will have different $brightness$, $fwhm$, $alpha$, $length$ and $speed$. 


\subsubsection{Clusters of moving objects}
Clusters are very similar to moving objects and have the same set of parameters. The only difference is that with moving objects when multiple objects are generated each object has different motion parameters ($speed$, $alpha$, and $length$). A cluster object allows multiple objects to have the same motion parameters and move the same way. The number of objects in one cluster is specified with the $objectCountPerCluster$ parameter and each object in the cluster has the same motion. The script supports the generation of multiple clusters with the parameter $count$.   

\subsubsection{Galaxies}
Similar to other objects, the script allows to generate multiple galaxies specifying their number by $count$. Elliptical galaxies have an inner core that is very bright, small, and concentrated. The outer part is larger in the area and the brightness is rapidly fading away from the core. The user can define the brightness of the inner core using the $brightness$ parameter. The brightness of the outer area is calculated using $brightnessFactor$ which defines the percentage of the $brightness$ and is used in the following manner: 

\begin{equation} \label{eq:brightnessGalaxy}
    b_a = brightnessFactor \cdot b_c
\end{equation}

where $b_a$ is the brightness of the outer area, and $b_c$ is the brightness of the core. 
Another parameters include $sigmaX$ and $sigmaY$ that define the variance of the outer area in x, y direction, and $sigmaFactor$ that describes the percentage of the variances for the inner core which is calculated as follows: 

\begin{equation} \label{eq:sigmaGalaxy}
    \begin{split}
        sigmaX_c = sigmaFactor \cdot sigmaX \\
        sigmaY_c = sigmaFactor \cdot sigmaY
    \end{split}
\end{equation}
where $sigmaX_c$, $sigmaY_c$ are variance of the inner core of the galaxy in the x,y direction. 
Lastly, the galaxy has its rotation which is defined by $alpha$ and contains values from 0 to 180 degrees. 

