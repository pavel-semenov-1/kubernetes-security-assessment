\section{Aggregator}
\label{sec:aggregator}

Aggregator service defines a common interface for all security scanner runners and provides a common interface for job tracking. Listing~\ref{lst:runner-interface} shows a code snippet of the definition.

\begin{lstlisting}[language=Go, caption={[A common inteface for runners] A common inteface for runners.}, label={lst:runner-interface}]
    /* Runner defines a common interface 
    for all security scanner runners */
    type Runner interface {
        Run() error
        GetStatus() JobStatus
        CleanUp() error
        Watch(*sql.DB) (int, string)
    }
\end{lstlisting}

Aggregator uses Kubernetes API to create and delete jobs and watch for their completion. Each runner is provided with the same Kubernetes clientset. It is a set of generated Go clients that allow us to interact with the Kubernetes API programmatically. It handles authentication, API requests, version negotiation, and resource management. Runner then defines a job as a Golang struct and sends it to the Kubernetes API for creation.

To extend Aggregator with a new scanner, the following steps should be considered:
\begin{enumerate}[noitemsep,nosep]
    \item If the scanner does not provide a Docker image, a custom Docker image should be defined and built.
    \item A new Runner implementation should be created. \lstinline{Run()} method should define and run the Kubernetes job for the scanner.
    \item Runner should be initiated inside the main Go function.
\end{enumerate}