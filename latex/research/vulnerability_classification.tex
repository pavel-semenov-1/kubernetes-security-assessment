\section{Security Threats Classification}
\label{sec:security-threats-classification}

We propose the following classification of the potential Kubernetes security risks. This is largely based on the security best practices gathered by Aquasec \cite{aquasec-security-best-practices,aquasec-kubernetes-vulnerability-database-misconfigurations}. We group related security threats into common subcategories and convert everything into the table format making it easier to navigate and study.

\begin{table}[H]
    \begin{center}
        \begin{tabular}{ | p{.27\textwidth} | p{.27\textwidth} | p{.40\textwidth} | } 
        \hline
        Category & Subcategory & Description \\ [0.5ex] 
        \hline\hline
        \multirow{5}{*}{} 1. Configuration Vulnerabilities & 1.1 Misconfigured RBAC  & Incorrectly set roles and permissions can lead to unauthorized access.  \\ \cline{2-3} 
                & 1.2 Pod Security Policies & Weak or missing pod security policies can allow privileged containers or insecure configurations.  \\ \cline{2-3} 
                & 1.3 Network Policies  & Inadequate network policies can expose services to unauthorized access.  \\ \cline{2-3} 
                & 1.4 Resource Limits  & Absence of resource limits (CPU, memory) can lead to resource exhaustion.  \\ \cline{2-3} 
                & 1.5 Preemption Policies & Undefined preemption policies/priorities. \\ \hline
        \multirow{3}{*}{} 2. Container Vulnerabilities & 2.1 Base Image Vulnerabilities &  Using container base images with known vulnerabilities. \\ \cline{2-3} 
                & 2.2 Outdated Packages & Containers running outdated software with known exploits. \\ \cline{2-3} 
                & 2.3 Exposed Secrets & Sensitive data (tokens, passwords) exposed in environment variables or volumes. \\ \hline
        \multirow{3}{*}{} 3. Kubernetes API Server Vulnerabilities & 3.1 API Server Exposure &  Unrestricted access to the Kubernetes API server. \\ \cline{2-3} 
                & 3.2 Audit Logging & Lack of or improperly configured audit logging that hinders incident detection and response. \\ \cline{2-3} 
                & 3.3 ETCD Data Exposure & Unsecured etcd exposing sensitive cluster data. \\ \hline
        \multirow{2}{*}{} 4. Network and Communication Vulnerabilities & 4.1 Service Exposure &  Services unnecessarily exposed to the internet. \\ \cline{2-3} 
                & 3.2 Ingress/Egress Controls & Inadequate controls over ingress and egress traffic. \\ \hline
        \multirow{2}{*}{} 5. Runtime and Execution Vulnerabilities & 5.1 Runtime Privileges &  Containers with root or elevated privileges. \\ \cline{2-3} 
                & 5.2 Containers with root or elevated privileges. & Usage of insecure system calls from within containers. \\ \hline
        \end{tabular}
    \end{center}
    \caption{Kubernetes security threat classification.}
    \label{tab:kubernetes-security-threat-classification}
\end{table}