\chapter{Astronomical data} \label{chap:astronomicaldata}

% nejaky obkec o tom ze data sa ziskavaju cez teleskopy a tie maju CCD chipy a ze snimky su vo formate FITS. a potom ze v tejto casti prejdeme rozne features ktore sa na snimkach nachadzaju, scenare a defekty a ako tieto nezaduce data odstranujeme
% FITS frames
% AGO data

In this section, we will first talk about how space debris images are acquired. We will describe the parameters of the telescope that captured the images used within this thesis. We will also define the format of the astronomical images. Next, we will explain what type of features can be present on the images and the defects and noises affecting them. In the last section, we present the image calibration process that is used to reduce the noises and defects to achieve clean images used for further processing and analysis. 

\section{AGO70}
The space debris images are acquired using astronomical telescopes. 
In our thesis, we are working with images captured at the Astronomical and Geophysical Observatory (AGO) in Modra. The acquisition of images at AGO is performed by the reflecting Newtonian telescope (AGO70) \cite{ago702018}, shown in the Figure \ref{img:ago70}. 

\begin{figure}[h]
    \centering
    \includegraphics[width=.5\textwidth]{images/ago70.png}
    \caption{The telescope AGO70 in Modra. Source: \cite{ago702018}.}
    \label{img:ago70}
\end{figure}

In 2016, after Slovakia became the 9th member of the ESA Plan for the European Cooperating States (PECS), a contract with the Faculty of Mathematics, Physics, and Informatics (FMPI) was signed. The first action was to transform the telescope in AGO from an amateur observation tool into a professional optical system used for regular tracking of space debris. Some examples of images acquired from the space debris observations performed by AGO70 are shown in the Figure \ref{fig:ago70images}. The installation of AGO70 finished in 2016 and the telescope has the following parameters: 

\begin{itemize}
    \item 700 mm primary parabolic mirror
    \item gravity actuator supporting the parabolic mirror
    \item focal length of 2962 mm
    \item FLI Proline PL1001 Grade 1 CCD camera
    \item 24 $\mu$m pixel size of the CCD camera
    \item resolution 1024x1024
    \item 28.5' x 28.5' field of view
    \item 16 bit per pixel images
\end{itemize}


\begin{figure}[!h]
    \begin{subfigure}{.3\textwidth}
        \centering
        \includegraphics[width=\textwidth]{images/StreakPoint00.jpg}
    \end{subfigure}
    \hfill
    \begin{subfigure}{.3\textwidth}
        \centering
        \includegraphics[width=\textwidth]{images/galaxypoints.jpg}
    \end{subfigure}
    \hfill
    \begin{subfigure}{.3\textwidth}
        \centering
        \includegraphics[width=\textwidth]{images/StreakStreak2.jpg}
    \end{subfigure}
    \hfill
    \caption{Examples of some images acquired by AGO70. }
    \label{fig:ago70images}
\end{figure}


%AGO70 has a very thin 700 mm primary parabolic mirror, which is supported by the gravity actuator. The mirror is placed at a focal length of 2962 mm. The telescope is equipped with FLI Proline PL1001 Grade 1 CCD camera and has a 28.5' x 28.5' field of view. Acquired images have a resolution of 1024x1024 pixels and contain values ranging from 0 to 65 535. 

\section{FITS format}
All images acquired from AGO70 are in the format of Flexible Image Transform System (FITS) files. It's an open standard digital format, which is very common for storing astronomical data. This data format will be used within this thesis. The structure of the FITS file is made up of two parts: header and data block \cite{fits}.

A header is a readable data structure that contains multiple keyword/value pairs. It is used for storing image metadata such as size, coordinates, and origin. With astronomical data, the header is very useful in providing photometric and spatial calibration information such as exposure time, readout noise, RADEC coordinates, etc \cite{fits}. An example of the FITS header of the image acquired by AGO70 is shown in the Figure \ref{img:fitsheader}. 

\begin{figure}[h]
    \centering
    \includegraphics[width=.9\textwidth]{images/fitsheader2.JPG}
    \caption{The FITS header of the image from AGO70.}
    \label{img:fitsheader}
\end{figure}

The data block of the FITS file can store an N-dimensional array of arbitrary size. The data array usually represents an array of image pixel values or tabular data \cite{fits}.
\section{Features} \label{sec:features}
Depending on the relative position and dynamics between the observer
and the object of interest, signals from RSOs can appear differently. The most common shape is a point and streak, which will be explained in greater detail in this section. 

Moreover, other objects like galaxies, nebulas, and comets are present in the universe and can also appear in images. Their profile is significantly different from points and streaks and has more diffuse features. However, for the purpose of this thesis, we will solely focus only on galaxies and more specifically - elliptical galaxies. 
% toto neviem ci tam budem davat lebo tam chcem opisat aj galaxie
%Note, that other types of features exist. Diffuse sources like galaxies or comet tails are less common but cannot be forgotten. However, for this thesis, they are not relevant and we will solely focus only on point-like and streak-like features.   

\subsection{Point}

The shape of the point source of the light on the CCD image is defined by the point spread function - PSF. As the light is passing through the atmosphere, the point on the CCD image is smeared. The smearing effect, also called seeing, is the dominant feature of the PSF. 
Under good optic and tracking conditions, PSF is usually circularly symmetric and can be approximated using a central Gaussian core. The measure used to express the angular size of the PSF is called FWHM - full width at half maximum \cite{romanishin2006introduction}. It measures the diameter of the Gaussian core in half of its maximum amplitude as can be seen in the Figure \ref{img:fwhm0}.  

\begin{figure}[h]
    \centering
    \includegraphics[width=0.5\textwidth]{images/fwhm.png}
    \caption{FWHM shown on the Gaussian curve. }
    \label{img:fwhm0}
\end{figure}

All objects that appear as points on the image, follow the PSF, and therefore all have the same FWHM and shape. However, brighter stars appear bigger on the image and this is caused by the intensity of the pixels belonging to the star \cite{romanishin2006introduction}. This phenomenon is explained in the Figure \ref{img:fwhm01}.


Another important feature of the PSF is that there is no defined edge. The intensity of the function is slowly fading until it blends with the background.


\begin{figure}[H]
    \centering
    \includegraphics[width=0.7\textwidth]{images/fwhmstars.JPG}
    \caption[Comparison of faint and bright star with the same FWHM and shape]
    {
    The image shows two stars with the same FWHM, but their appearance on the frame (black dots on top of the image) differs significantly. The bright star (right) has flux five times bigger than the faint star (left). 
    On the plot in the bottom part of the image, we can observe the dashed line, which marks 1800 counts/pixel. This means that pixels below, with low intensity, are considered dark, while pixels above are considered brighter or white.  
    First, let's consider a faint star (left) which consists of a few bright pixels, but overall has a very low intensity of pixels. Low-intensity pixels appear black as they are below the dashed line. On the image, the faint star will appear smaller than the brighter star, because only the peak of the PSF will be distinguishable from the dark background. 
    Now consider a bright star (right), which has an overall very high intensity. On the image, the bright star will appear much bigger, because the majority of the PSF is above the dashed line.  
    
    Source: \cite{romanishin2006introduction}.
    
    }
    \label{img:fwhm01}
\end{figure}


\subsection{Streak}

A streak on the image, which resembles a line, can be described by its length, orientation, brightness, and width. 
Streak is approximated using multiple point-spread functions (Gaussian PSFs) moving at a constant rate in one direction and forming a line. PSFs are connected and also overlap each other to form a flat top of the streak-like shape. This function is referred to as PSF-Convolution Trail Function. % citacia

Let's situate a coordinate system on the streak. The direction in which the streak has the highest variance is the $x'$ axis and perpendicular to it is the $y'$ axis. This coordinate system is not consistent with the $(x,y)$ coordinate system of the image. This is because the streak has its orientation $\theta$ and doesn't necessarily need to be aligned with the image coordinate system. 
In this coordinate system, the length of the streak $L$ is measured on the $x'$ axis at the half-height of the function. The projection of the streak signal on the $y'$ axis creates a perpendicular profile of the streak and the width $\sigma$ of the PSF is measured at half-width. This is illustrated in the Figure \ref{img:line0}. 
The flux $f$ of the streak at any point situated in the new orthogonal coordinate system $(x',y')$ can be expressed as:

\begin{equation}
    f(x',y') = b(x',y') + \frac{\Phi}{L} \cdot \frac{1}{\sqrt{2 \pi \sigma^2}} \cdot \int_{-L/2}^{+L/2} 
    exp \left(  
        - \frac{1}{2 \sigma^2} \cdot \left[ (x' - l)^2 + (y')^2 \right]
    \right) \,dl
\end{equation}

where $\Phi$ is the total photometric flux in the streak and $b(x',y')$ is the background flux at the same point \cite{thesisNagy}. 


\begin{figure}[h]
    \centering
    \includegraphics[width=0.7\textwidth]{images/line.png}
    \caption[Length and width of the streak on the image]
    {Length and width of the streak explained on the image.
    Image source: \cite{articleStreaks}.}
    \label{img:line0}
\end{figure}

\subsection{Galaxy}
According to \cite{hubble} galaxies can be classified into 3 categories based on their visual appearance:
\begin{itemize}
    \item Elliptical 
    \item Spiral 
    \item Lenticular
\end{itemize}

In this thesis, we will only focus on the elliptical galaxies, which in the case of modeling and using ellipticity of 1 can also be a good approximation of spiral and lenticular galaxy shapes to be used for the training purposes.

The main characteristic of an elliptical galaxy is that its profile resembles ellipses on the images. They are highly concentrated and the light going from the center fades smoothly and rapidly away. This creates a smooth diffuse profile with an undefined edge. Another interesting feature of elliptical galaxies is that with increasing exposure time of the image, the diameter of the galaxy is steadily increasing but the shape stays roughly the same \cite{hubble}.

Elliptical galaxies are denoted with the letter "E". In full notation "E" is followed by a number representing their degree of ellipticity, which is defined as follows: 

\begin{equation}
E = \frac{(a-b)}{a}
\end{equation}

where $a$ is the major diameter and $b$ is the minor \cite{hubble}. 

\section{Scenarios} \label{sec:scenarios}

In this section, we will explain different scenarios happening during the image capture, that affect how the object appears in the image. Scenarios depend on the mode of tracking (sidereal, object) and the relative velocity between the moving object of interest and the telescope. 

\subsection{Point-like stars, point-like objects}
This is a typical scenario occurring in astronomical images and is shown in the Figure \ref{fig:pointpoint1}. The angular velocity of the moving object of interest is so small that during the exposure time their position on the image doesn't seem to move. The speed is so slow that they don't cross more than one pixel, which leads to the appearance of the object as a point. This applies to both, objects of interest as well as stars. 

This scenario commonly happens in an asteroid field, small solar body system field, and can also appear in space-debris observations when high cadence and low exposure time is used. 

\begin{figure}[!h]
    \centering
    \includegraphics[width=.4\textwidth]{images/PointPoint.png}
    \caption{An example of point-point scenario.}
    \label{fig:pointpoint1}
\end{figure}

\subsection{Point-like stars, streak-like objects}
In ground-based images, stars appear as points, due to slow relative dynamics between the observer. If the exposure time on the image was longer stars would appear as streaks too, as a result of the motion of the Earth. 
During sidereal tracking, the telescope is moving to compensate for the Earth's motion and this results in point-like stars remaining in the same place on images.
In space-based images, stars appear as points if the camera is fixed.


Regarding moving objects, if the exposure time is long enough that the object is crossing more than one pixel, they appear as streaks. This commonly happens when the object is moving with high velocity. %If the exposure time is long compared to the angular velocity, it leads to objects appearing as streaks. 

The point-like stars and streak-like object scenario can be observed in the series of images depicted in the Figure \ref{fig:pointstreak0}.

%However if the exposure time is too short for an object to cross more than one pixel it would appear as a point, which is the point-point scenario explained above. 

\begin{figure}[!h]
    \begin{subfigure}{.3\textwidth}
        \centering
        \includegraphics[width=\textwidth]{images/PointStreak1.png}
        \label{fig:pointstreak1}
    \end{subfigure}
    \hfill
    \begin{subfigure}{.3\textwidth}
        \centering
        \includegraphics[width=\textwidth]{images/PointStreak2.png}
        \label{fig:pointstreak2}
    \end{subfigure}
    \hfill
    \begin{subfigure}{.3\textwidth}
        \centering
        \includegraphics[width=\textwidth]{images/PointStreak3.png}
        \label{fig:pointstreak3}
    \end{subfigure}
    \hfill
    \caption{An example of series of images capturing a streak-like object and point-like stars. }
    \label{fig:pointstreak0}
\end{figure}

\subsection{Streak-like stars, point-like objects}
When the telescope is in object tracking mode, the focus is aimed at the moving object. Thus telescope is following the object matching its speed and direction. The moving object, therefore, appears as a point while stars appear as streaks. All stars will have the same length and direction of the streak-like shape. An example of one point-like object and otherwise streak-like stars are shown in the Figure \ref{fig:streakpoint0}. 

If there are more moving objects present on the frame, they can either appear as points or streaks. In case the other moving objects are moving at a similar speed and direction as the tracked object, they will also appear as points. This scenario happens when a telescope is tracking a cluster of objects. Otherwise, if other objects have different speeds and directions, they will appear as streaks but with different lengths and directions as streaks created by stars. 
This applies to both, space- and ground-based images. All these effects can influence the performance of segmentation or recognition algorithms.

\begin{figure}[!h]
    \begin{subfigure}{.3\textwidth}
        \centering
        \includegraphics[width=\textwidth]{images/StreakPoint1.png}
        \label{fig:streakpoint1}
    \end{subfigure}
    \hfill
    \begin{subfigure}{.3\textwidth}
        \centering
        \includegraphics[width=\textwidth]{images/StreakPoint2.png}
        \label{fig:streakpoint2}
    \end{subfigure}
    \hfill
    \begin{subfigure}{.3\textwidth}
        \centering
        \includegraphics[width=\textwidth]{images/StreakPoint3.png}
        \label{fig:streakpoint3}
    \end{subfigure}
    \hfill
    \caption{An example of series of images capturing a streak-like stars and one point-like object. }
    \label{fig:streakpoint0}
\end{figure}

\subsection{Streak-like stars, streak-like objects}
This is a common scenario happening during the survey of the sky, when stars nor objects are being tracked, which is depicted in the Figure \ref{fig:streakstreak0}. When taking an image of the sky field during the survey, unknown objects can appear randomly, leaving streak-like features of different lengths and directions on the image. 
Stars appear as streaks in ground-based observations due to the motion of Earth and long exposure time. However streak-like stars can also be caused by the motion of the telescope in both ground and space-based observations. 

\begin{figure}[!h]
    \begin{subfigure}{.3\textwidth}
        \centering
        \includegraphics[width=\textwidth]{images/StreakStreak1.png}
        \label{fig:streakstreak1}
    \end{subfigure}
    \hfill
    \begin{subfigure}{.3\textwidth}
        \centering
        \includegraphics[width=\textwidth]{images/StreakStreak2.png}
        \label{fig:streakstreak2}
    \end{subfigure}
    \hfill
    \begin{subfigure}{.3\textwidth}
        \centering
        \includegraphics[width=\textwidth]{images/StreakStreak3.png}
        \label{fig:streakstreak3}
    \end{subfigure}
    \hfill
    \caption{An example of series of images capturing a streak-like stars and one streak-like object. }
    \label{fig:streakstreak0}
\end{figure}

\section{Defects and noises} \label{sec:defects}

There are various ways the CCD image can be distorted or corrupted and also numerous sources of these defects. 


\subsection{Statistical noise}

    %%%%%%%%%%%%%%%%%%%%%%%%%%%% 
    \subsubsection{Photon noise}
    
    The distribution of photons falling on the CCD chip obeys simple counting statistics. 
    Let's assume the CCD chip is illuminated with constant uniform light and that the sensitivity of each pixel is the same. 
    For each finite moment, there is an amount of photons detected on the chip which varies.
    If we were to plot the amount of photons detected and the occurrences of each amount in a histogram, we would quickly realise that the distribution is following a Poisson distribution. As we are dealing with large amount of photons, Poisson distribution can be approximated with the Gaussian function: 
    
    \begin{equation}
     \label{eqn:gaussian}
        g(x) = \frac{1}{\sigma \cdot \sqrt{2 \pi}} \cdot exp \left(  - \frac{1}{2} \cdot \frac{(x - \mu)^2}{\sigma^2} \right)   
    \end{equation}
    
    where $\mu$ is the average or the most common amount of photons detected and $\sigma$ is the standard deviation. 
    The number of photons detected on the chip can also be referred to as a signal $S$ \cite{romanishin2006introduction}.
    
    Let's denote, that $S_{star}$ is the signal from the star and $S_{sky}$ is the signal from the sky. From \cite{romanishin2006introduction} the noise from the star $\sigma_{star}$ and the noise from the sky $\sigma_{sky}$ can be expressed as:
    
    \begin{equation}
    \label{eqn:starskynoise}
        \sigma_{star} = \sqrt{S_{star}} \quad \sigma_{sky} = \sqrt{S_{sky}}
    \end{equation} 

    and addition of these noises is done in a following manner:
    \begin{equation}
    \label{eqn:additionNoises}
        \sigma = \sqrt{\sigma_{star}^2 + \sigma_{sky}^2}
    \end{equation}

\subsection{Internal}

Internal defects and noises are caused by issues within the CCD chip. This includes mechanical and electrical issues. 
 

    %%%%%%%%%%%%%%%%%%%%%%%%%%%% 
    \subsubsection{Bias}
    
    In the process of accumulating photo-electrons in the CCD chip, let's suppose that each pixel accumulated 2 photo-electrons on average. Given how the analog to digital conversion works, a number of counts $N_{counts}$ is proportional to a number of photo-electrons $N_{elec}$ accumulated in the pixel and is expressed by the following: 
        
    \begin{equation}
    \label{eqn:bias}
        N_{elec}  = gain \cdot N_{counts}
    \end{equation}
    
    For the simplicity of the example let's suppose that the gain is 1 e-/ADU. According to the formula, each pixel should have 2 counts, but due to the readout noise, the number of counts will follow the Gaussian distribution. The mean of the Gaussian is 2 counts, with a standard deviation given by the readout noise, which could be as much as 3 counts. This would mean that the number of counts in some pixels would be negative. However, the ADC cannot represent negative numbers, which means that the data would be corrupted \cite{articleCCDartifacts} \cite{phy217}.
    To fix this issue, bias voltage was applied to the CCD detector. It is a constant offset voltage, which means that even if the pixel contained no photo-electrons, the ADC returns a high positive value. This solves the issue with negative values. The constant bias voltage is identified through the BIAS frame \cite{articleCCDartifacts} \cite{phy217}.
    
        
 
    %%%%%%%%%%%%%%%%%%%%%%%%%%%% 
    \subsubsection{Readout noise}
    
    Reading out the image from the CCD chip includes charge shifting through the chip followed by conversion to the voltage. Due to the limitations in the current technology, the voltage in the pixel cannot be measured perfectly and readout noise is introduced. 
    The noise is expressed as the number of electrons per pixel. This number is characteristic for each CCD chip and is usually defined by manufacturers.
    Readout noise is generated during the process of reading the image, therefore it is independent of the exposure time of the image. However, the amount of noise increases with the higher speed of the readout \cite{articleCcdOnline}.
    
    From \cite{bolte15} readout noise in the aperture can be expressed as: 
    
    \begin{equation}
    \label{eqn:readoutNoise}
         \sigma_{RN} = \sqrt{N_{pix} \cdot {RN}^2 }
    \end{equation}
    
    where $N_{pix}$ is a number of pixels in the aperture and $RN$ is readout noise in electrons per pixel.

    Note, that the readout noise is not following Poisson distribution and is not related to the signal coming from stars or sky \cite{matfyzpress01}. Therefore Formula \ref{eqn:starskynoise} cannot be used, when expressing the readout noise. 

    %%%%%%%%%%%%%%%%%%%%%%%%%%%% 
    \subsubsection{Dark current}
    
    Photo-electrons in CCD pixels are produced by photons from incoming light. However, thermal excitation can also produce electrons in pixels, and it's impossible to differentiate between them and photo-electrons coming from the light \cite{phy217}.
    This effect is called dark current and is expressed in electrons per second per pixel at the defined temperature. The amount of dark current is linearly proportional to chip temperature and exposure time. To minimize the effect, CCD chips are usually cooled down to a temperature below 0°C \cite{articleCcdOnline}.
    
    The amount of dark current is also affected by pixel size, chip architecture, and production technology. Therefore manufacturers usually define the amount of dark current for specific models of CCD chips.
    Even after the chip is cooled, the amount of dark current is not negligible. To detect the dark current in the image, DARK frames were created \cite{articleCcdOnline}. The noise from the dark current is also following Poisson distribution \cite{matfyzpress01}. 
    
    From \cite{bolte15} the noise caused by dark current in our aperture can be expressed as:
    
    \begin{equation}
    \label{eqn:darkcurrent}
        \sigma_{DC} = \sqrt{D \cdot N_{pix} \cdot t}
    \end{equation}

    
    where $D$ is dark current in $e^-/second/pixel$, $N_{pix}$ is number of pixels in the aperture and $t$ is the exposure time.
    
    
    \begin{figure}[!h]
    \centering
        \begin{subfigure}{.3\textwidth}
        \centering
            \includegraphics[width=\textwidth]{images/hotpixels.png}
            \caption{Hot pixels.}
            \label{fig:hotpixels}
        \end{subfigure}
        \begin{subfigure}{.3\textwidth}
            \centering
            \includegraphics[width=\textwidth]{images/deadcolumns.jpg}
            \caption{Dead columns.}
            \label{fig:deadcolumns}
        \end{subfigure}
        
        \vspace*{4mm}
        
        \begin{subfigure}{.3\textwidth}
            \centering
            \includegraphics[width=\textwidth]{images/traps2.jpg}
            \caption{Traps.}
            \label{fig:trap}
        \end{subfigure}
        \begin{subfigure}{.3\textwidth}
            \centering
            \includegraphics[width=\textwidth]{images/saturationtrail.png}
            \caption{Saturation trail.}
            \label{fig:saturationtrail}
        \end{subfigure}
        \caption{Examples of some internal defects present on FITS images acquired by AGO70.}
        \label{fig:internaldefects}
    \end{figure}
    
    
    
    
    
    %%%%%%%%%%%%%%%%%%%%%%%%%%%% 
    \subsubsection{Hot Pixels}
    
    Hot pixels (Figure \ref{fig:hotpixels}) are individual pixels that have a very high dark current, meaning they are really bright or fully saturated. This is related to the defect of the chip which was either created during the manufacturing or by aging. Fully saturated pixels have reached their limit of detected photo-electrons, which means that any incoming photons will not generate any electrons. 
    As this is the result of the CCD chip issue, their positions on the image are fixed. This makes them easy to detect on a DARK frame.
    % citacia

        
    %%%%%%%%%%%%%%%%%%%%%%%%%%%% 
    \subsubsection{Dead Pixels}
    
    Dead pixels are also individual pixels, which on the contrary are not sensitive to any light and are completely dark. Similar to hot pixels, their positions are fixed from image to image and can be detected on the DARK frame.
    % citacia
    
    %%%%%%%%%%%%%%%%%%%%%%%%%%%% 
    \subsubsection{Dead columns} 
    
    Dead columns (Figure \ref{fig:deadcolumns})are columns of pixels corrupted in some way. %as depicted in the figure \ref{fig:deadcolumns}. 
    They can be either completely dark (not sensitive to any incoming photons) or fully saturated (reached the limit of detected photo-electrons). Apart from that, dead columns can also have a constant value added to what their real value should be. 
    They are various reasons for this defect, mechanical or electronical \cite{articleCCDartifacts}.
    
    
    
    %%%%%%%%%%%%%%%%%%%%%%%%%%%% 
    \subsubsection{Traps} 
    
    Traps (Figure \ref{fig:trap}) occur when some pixels on the CCD chip are damaged in a way that they cannot transfer photo-electrons \cite{articleCCDartifacts}. 
    As mentioned earlier during the process of reading the image, the charge is shifted through pixels. All rows are shifted one row below and the last row is shifted to the horizontal register from which the charge is shifted to the ADC \cite{articleCcdOnline}. However, if a pixel is damaged, the charge cannot pass through during the shifting of the rows. 
    
    Let's assume, the damaged pixel is in the first column and on the 4th-row counting from the top. Note that damage is only present in the first column, which contained the damaged pixel and other columns are unaffected.
    
    Pixels below the 4th row are read normally. However, pixels in the first three rows cannot be shifted through the damaged 4th pixel. Electrons in these pixels are therefore trapped and their charge cannot be converted to counts. This will result in completely dark pixels in these first three rows. %An example of a vertical trap is shown in the figure \ref{fig:trap} 
    
    Note, that traps are vertical when the CCD chip is read horizontally, by shifting charge from one row to another. However, in case the chip is read vertically and the charge is shifting through columns, traps will be horizontal. 


    %%%%%%%%%%%%%%%%%%%%%%%%%%%%
    \subsubsection{Saturation trail}
    
    Each pixel on the CCD chip has a certain amount of photo-electrons it can store. Some very bright sources can illuminate pixels to the point where they are completely filled up. As the pixel is filled to its limit, photo-electrons from the saturated pixel can start leaking to adjacent pixels. These extra electrons create a trail of saturated pixels - saturation trail (Figure \ref{fig:saturationtrail}) \cite{articleCCDartifacts}.
    
    Similar to traps, trails are either horizontal or vertical and it depends on the way the CCD chip is read. 
    



\subsection{External}

External noises and defects are caused by external forces such as moonlight, sunlight, particles in the universe, the atmosphere, and many others. All these effects are discussed below.

    
    %%%%%%%%%%%%%%%%%%%%%%%%%%%% 
    \subsubsection{Sky background noise} 
    
    One of the most dominant sources of noise in CCD images is the noise from the sky. This is caused by various sources of light present in the sky as well as some photons flying through space. 
    %For stars to be seen, their signal must be brighter than the noise coming from the sky. However, this does not mean that fainter stars will never be seen. The amount of signal is not as important as the ratio of the signal to the current noise in the sky. As this is a common problem in astronomy, this ratio is well known and is called signal-to-noise ratio - SNR. The higher the ratio, the easier it is to see the signal from the stars.
    The sky noise is not constant and changes depending on various circumstances. As the sky gets brighter, the noise from the sky increases as well. This is caused by light pollution. Decreasing noise can be therefore observed when the sky is the darkest. Since the signal from stars has a constant value and the noise has a decreasing tendency this results in higher SNR of star signals \cite{romanishin2006introduction}. 
    
    Sky observations are usually done when the sky is darkest and at remote places to avoid light pollution. However even this does not guarantee the perfect dark sky. Various atomic processes are happening in the atmosphere. Molecules and atoms are constantly colliding with each other, which causes the excitation of atoms. This causes the air to glow which is yet another source of light contributing to the sky background noise. 
    In the case of space-based observations, the sky is darker compared to observations from Earth as there is no light coming from artificial sources and the atmosphere. However, another dominant source of light is present, which is called zodiacal light. This is sunlight scattered by dust particles and is present in the whole Solar system \cite{romanishin2006introduction}.
    
    %With low exposure time, the amount of sky noise is also very low and electronic noises are dominating the overall noise.
    
    %%%%%%%%%%%%%%%%%%%%%%%%%%%% 
    \subsubsection{Stray light} 

    Stray light (Figure \ref{fig:straylight})is caused by the reflection of light from optical surfaces like tubes or domes. Various sources of light can be reflected. This includes scattered moonlight which causes a bright gradient present in the image. Artificial sources of light such as diodes or internal lights can also be reflected on the image.
    % citacia
    
    %%%%%%%%%%%%%%%%%%%%%%%%%%%% 
    \subsubsection{Diffuse sources} 
    
    Diffuse sources such as galaxies (Figure \ref{fig:diffgalaxy}), nebulas, and comets (Figure \ref{fig:comet}) are also contributing to image corruption. They are not explicitly defects or noises, but their presence in the image can cause problems during photometry as they are contributing to the sky‘s brightness. 

        
    %%%%%%%%%%%%%%%%%%%%%%%%%%%%
    \subsubsection{Dust on filter}
    
    Dust particles (Figure \ref{fig:dust}) present on the lens or filters of the aperture can also cause problems in the image. Apart from other defects mentioned, which are usually only locally on the image, dust particles affect the image globally. The particles cast shadows on the detector and this manifests on the image in a form of dark rings \cite{articleCCDartifacts}.
    FLAT FIELD frame can detect these defects and thus are later corrected in a photometric reduction \cite{articleParimucha}.
    
    
    \begin{figure}[!h]
    \centering
        \begin{subfigure}{.3\textwidth}
        \centering
            \includegraphics[width=\textwidth]{images/straylight.png}
            \caption{Stray light.}
            \label{fig:straylight}
        \end{subfigure}
        \begin{subfigure}{.3\textwidth}
            \centering
            \includegraphics[width=\textwidth]{images/galaxyreal.jpg}
            \caption{Galaxy.}
            \label{fig:diffgalaxy}
        \end{subfigure}

        \vspace*{4mm}

        \begin{subfigure}{.3\textwidth}
            \centering
            \includegraphics[width=\textwidth]{images/comet.jpg}
            \caption{Comet.}
            \label{fig:comet}
        \end{subfigure}
        \begin{subfigure}{.3\textwidth}
            \centering
            \includegraphics[width=\textwidth]{images/dust.jpg}
            \caption{Dust ring.}
            \label{fig:dust}
        \end{subfigure}
        \caption{Examples of some external defects present on FITS images acquired by AGO70.}
        \label{fig:externaldefects}
    \end{figure}
 
    %%%%%%%%%%%%%%%%%%%%%%%%%%%% 
    \subsubsection{Cosmic rays}
    
    %A CCD chip is an instrument detecting photons emitting from light sources. Each photon creates a single electron in a specific pixel of the chip. 
    Flying through space are also high-energy particles called cosmic rays.  Each particle can excite hundreds or thousands of electrons, which can also cross through multiple pixels. 
    A cosmic ray hitting the CCD chip is an unpredictable event that can occur randomly affecting the image locally \cite{imageProc}. The temperature of the chip or its defects do not affect the event of cosmic rays hitting the chip. However, the longer the exposition time, the more cosmic rays hit the chip. 
    Cosmic rays can manifest as a single very bright pixel but sometimes it affects several adjacent pixels. The created object has strong and asymmetric features with sharp edges \cite{irafArticle}.
    
    Their profile can be classified into three categories \cite{inbookCosmics} which are shown in the Figure \ref{img:cosmicraysreal}  
        \begin{itemize}
            \item spot-like - resemble dots
            \item track-like - resemble straight lines
            \item worm-like - resemble polylines and curves
        \end{itemize}
    
    
    They could also be described as a group of connected pixels, which have count values higher than the background and at least one pixel having a significantly high value \cite{inbookCosmics}.
    
    When it comes to ground-based observations, they don't suffer from cosmic rays as much as space-based. Mainly because the amount of cosmic rays significantly increases in space. 

    \begin{figure}[!h]
    \centering
        
        \begin{subfigure}{\textwidth}
            \centering
            \includegraphics[width=.13\textwidth]{images/spot1.png}
            \includegraphics[width=.13\textwidth]{images/spot2.png}
            \includegraphics[width=.13\textwidth]{images/spot3.png}
            \includegraphics[width=.13\textwidth]{images/spot4.png}
            \caption{Spots.}
        \end{subfigure}
        \vskip\baselineskip
        \begin{subfigure}{\textwidth}
            \centering
            \includegraphics[width=.13\textwidth]{images/track1.png}
            \includegraphics[width=.13\textwidth]{images/track2.png}
            \includegraphics[width=.13\textwidth]{images/track3.png}
            \includegraphics[width=.13\textwidth]{images/track4.png}
            \caption{Tracks.}
        \end{subfigure}
        \vskip\baselineskip
        \begin{subfigure}{\textwidth}
            \centering
            \includegraphics[width=.13\textwidth]{images/worm1.png}
            \includegraphics[width=.13\textwidth]{images/worm2.png}
            \includegraphics[width=.13\textwidth]{images/worm3.png}
            \includegraphics[width=.13\textwidth]{images/worm4.png}
            \caption{Worms.}
        \end{subfigure}
        
        \caption{Examples of three categories of cosmic rays. Source \cite{cosmicrayimageall}.}
        \label{img:cosmicraysreal}  
    \end{figure}
\section{Photometric reduction of CCD images} \label{sec:photoreduction}


An image obtained by a telescope using a CCD camera is called a raw image. A raw image is affected by before mentioned defects and noises. Therefore, the image needs to be corrected to a certain level to obtain precise photometric results \cite{articleParimucha}. This process is called image calibration, which contains several steps discussed here below.  

\subsection{Calibration images}
The image calibration consists of three steps: subtraction of BIAS image, subtraction of DARK image and division by FLAT FIELD image.

    %%%%%%%%%%%%%%%%%%%%%%%%%%%% 
    \subsubsection{BIAS frame}
    As mentioned earlier, to solve an issue with negative values in ADC, bias voltage was introduced.
    However, during photometry, the offset bias value needs to be subtracted from the image to achieve correct values. To retrieve the bias values on the pixels, the BIAS frame is created. 
    The aim is to retrieve an image without any residual signal in the pixels originating from the camera. This is achieved by taking an image with a closed shutter and zero exposure time \cite{articleParimucha}.
     
             
    %%%%%%%%%%%%%%%%%%%%%%%%%%%%   
    \subsubsection{DARK frame} 

    DARK frames are introduced to detect dark current in the image. Apart from that, they are also capable of detecting hot and dead pixels on the image. 
    To create a DARK frame, an image is taken with a closed shutter to eliminate photo-electrons from stars and the sky. Values in the pixels are from the dark current and bias voltage. Therefore to obtain a correct DARK frame, bias values need to be subtracted from the DARK frame. 
    As mentioned before, dark current is proportional to chip temperature and exposure time. This implies that in order to have correct dark current values, the DARK frame needs to be taken with the same exposure time as the raw image. The same goes for the temperature of the CCD chip \cite{articleParimucha} \cite{articleCcdOnline} \cite{phy217}. 
    

    %%%%%%%%%%%%%%%%%%%%%%%%%%%%  
    \subsubsection{FLAT FIELD frame} 
    
    Taking an image of the perfectly uniform light source, would not result in an image with the same amount of counts in each pixel. Readout noise, bias voltage, and dark current are all contributing to this fact. However, even without them, the image would not be uniform. The main reason is the sensitivity of each pixel. Due to the manufacturing limitations, no two pixels convert the light photons into electrons the same way \cite{articleCCDartifacts} \cite{phy217}.
    
    The solution to this problem is by taking a FLAT FIELD frame, which corrects pixel to pixel sensitivity. As mentioned before, the frame is created by taking a picture of an evenly illuminated field. One of the most common ways is to take an image of the twilight sky.
    Similarly, as with the DARK frame, this FLAT FIELD frame contains values from bias voltage as well as dark current. To get a single correct FLAT FIELD frame, these values need to be subtracted.
    Another great feature of the FLAT FIELD frame is that it can also detect dust particles on the filters and lenses. These manifest as dark rings on the image. They are the same with different exposure times but vary from filter to filter. However, they also differ from time to time as some particles can be shifted, eliminated, or added. 
    Furthermore, flat fielding can clear dimming on the edge of the image. This is also called vignetting and is caused by out-of-focus obstructions in the light path \cite{articleParimucha} \cite{phy217}.
    
    After taking the FLAT FIELD frame, signal values are arbitrary, since it only means how bright the source of illumination was. The important part is the differences in the signal across the chip. FLAT FIELD frame is therefore normalized in a way that the average signal in each pixel is 1 \cite{phy217}.
    %% najst citaciu tohto
    
        
    %%%%%%%%%%%%%%%%%%%%%%%%%%%%  
    \subsubsection{MASTER frame}
    
    Due to the nature of reading the image from the CCD chip, every calibration image contains readout noise, which can cause issues during the calibration process. To minimize the noise, multiple calibration images are taken, which are then reduced to one MASTER frame. 
    The MASTER frame is created by taking average or median values of the pixel from the multiple calibration images (BIAS, DARK, FLAT FIELD).
    After the process, we have MASTER BIAS, MASTER DARK, and MASTER FLAT FIELD frames (shown in the Figure \ref{fig:masterframes}), and these are later used in photometric reduction \cite{articleParimucha}.
   
   
   
    \begin{figure}[!h]
    \centering
        \begin{subfigure}[t]{.3\textwidth}
            \centering
            \includegraphics[width=\textwidth]{images/biasframe.jpg}
            \caption{MASTER BIAS frame.}
            \label{fig:biasframe}
        \end{subfigure}
        \hfill
        \begin{subfigure}[t]{.3\textwidth}
            \centering
            \includegraphics[width=\textwidth]{images/dark90s.jpg}
            \caption{MASTER DARK frame with exposition time of 90 seconds.}
            \label{fig:darkframe}
        \end{subfigure}
        \hfill
        \begin{subfigure}[t]{.3\textwidth}
            \centering
            \includegraphics[width=\textwidth]{images/flatframe.jpg}
            \caption{MASTER FLAT FIELD frame.}
            \label{fig:flatframe}
        \end{subfigure}
        \hfill
        \caption{Examples of MASTER frames acquired by AGO in Modra.}
        \label{fig:masterframes}
    \end{figure}



\subsection{Formal definition}

The intensity of the raw image at the pixel $(x,y)$, with exposure time $t$ and temperature of CCD chip $T$ can be generally written as
\[ I(x,y,t,T) = b(x,y,T) + d(x,y,t,T) + i(x,y,t,T) f(x,y,t_f,T) \]

where $b(x,y,T)$ is the intensity of BIAS frame. Exposure time is ommited since the BIAS frame is taken with exposure time of zero. Intensity of DARK frame is denoted as $d(x,y,t,T)$. $f(x,y,t_f,T)$ is the response factor of FLAT FIELD frame, taken with exposure time $t_f$ \cite{articleParimucha}.

The intensity of the real object, which we want to obtain is denoted as $i(x,y,t,T)$. 
All the images are dependent on temperature $T$, which can be omitted from the equation since the CCD chip is cooled down and images are usually taken with the same temperature \cite{articleParimucha}.

To get the real intensity of the object the previous equation becomes
\[
i(x,y,t) = \frac{ I(x,y,t) - b(x,y) - d(x,y,t)}{f(x,y,t_f)}
\]

