Kubernetes v posledných rokoch rýchlo získava na popularite, pretože čoraz viac spoločností hľadá spôsoby, ako zvýšiť efektivitu vývoja a znížiť náklady na vývoj. Táto zvýšená popularita so sebou prináša väčšie vystavenie kybernetickým útokom a zvýšené obavy zainteresovaných strán o bezpečnosť Kubernetes.
Cieľom práce je porovnať a zhodnotiť moderné nástroje určené na odhaľovanie zraniteľností týkajúcich sa konfigurácie klastra, bežiacich podov alebo aj samotného klastra. Posúdenie bude prebiehať na lokálnom klastri s prednasadenými viacerými zraniteľnosťami, ako aj v reálnej podnikovej cloudovej infraštruktúre. Na základe výsledkov hodnotenia máme v úmysle buď vylepšiť niektorý z existujúcich nástrojov, alebo vyvinúť vlastný bezpečnostný rámec pre Kubernetes, ktorý bude schopný poskytnúť lepšie výsledky pri riešení klastrovej bezpečnosti.