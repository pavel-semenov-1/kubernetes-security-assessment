\section{Summary}

The summary of all three models is depicted in the Table \ref{tab:summary}, which contains the accuracy, recall and precision on the testing dataset. In both fine-tuned models we can see a considerable improvement from the model that was trained only on the synthetic data. According to this, we can clearly state that the incorporation of the real data, even in such a small amount, has enhanced the capability of our model. 

If we compare the two fine-tuning approaches, they performed just slightly different. However fine-tuning only FC layers have achieved higher accuracy and training the model took significantly less time. One epoch of training all the layers on the whole merged training dataset took approximately 4 minutes. On the other hand, one epoch of tuning FC layers with the training dataset containing only real images took less than 30 seconds.

{\renewcommand{\arraystretch}{1.4}
\begin{table}[h]
\centering
\begin{tabular}{|l|c|c|c|}
\hline
\textbf{Model} & \multicolumn{1}{l|}{\textbf{Accuracy}} & \multicolumn{1}{l|}{\textbf{Precision}} & \multicolumn{1}{l|}{\textbf{Recall}} \\ \hline
\textit{\begin{tabular}[c]{@{}l@{}}model trained \\ on synthetic data\end{tabular}} & 73.00 \% & 73.52 \% & 73.00 \% \\ \hline
\textit{\begin{tabular}[c]{@{}l@{}}model trained\\ on merged data\end{tabular}} & 87.00 \% & 87.15 \% & 87.00 \% \\ \hline
\textit{\begin{tabular}[c]{@{}l@{}}model with fine-tuned \\ FC layers\end{tabular}} & 89.00 \% & 89.19 \% & 89.00 \% \\ \hline
\end{tabular}
\caption{A summary of the performance on the testing dataset on all three selected models. }
\label{tab:summary}
\end{table}
}