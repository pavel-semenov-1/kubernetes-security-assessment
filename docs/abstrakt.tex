Zvýšený záujem o vesmírne aktivity spôsobil vznik vesmírneho odpadu obiehajúceho okolo Zeme. V posledných rokoch však vesmírny odpad predstavuje veľmi nebezpečný problém, ktorý môže ľahko ohroziť budúce vesmírne misie. Aby sa predišlo tomuto problému, je nevyhnutné pravidelné sledovanie a detekcia vesmírneho odpadu. Snímky získané z astronomických pozorovaní vesmírneho odpadu je potrebné spracovať a analyzovať, aby sme identifikovali astronomické objekty. Snímky zachytávajú signály z rôznych zdrojov. Od šumu spôsobeného nedokonalosťami CCD čipu, cez defekty spôsobené vonkajšími zdrojmi, alebo pozadím oblohy až po skutočné astronomické objekty, ako sú hviezdy, galaxie alebo objekty slnečnej sústavy (vesmírny odpad, satelity, kométy atď.)
Na správnu identifikáciu objektov, je potrebné obrázky najskôr očistiť od nežiadúcich efektov alebo navrhovaný algoritmus musí byť dostatočne robustný, aby sa zameral iba na signály zo skutočných astronomických objektov.

Na základe rozsiahleho výskumu analyzujúceho rôzne existujúce riešenia sme sa rozhodli použiť konvolučnú neurónovú sieť na vyriešenie úlohy klasifikácie vesmírnych objektov. V našej práci sme navrhli architektúru siete a prešli rozsiahlym procesom ladenia hyperparametrov. Aby sme sieť natrénovali a zároveň zabezpečili, že dáta sú robustné a vyvážené, generujeme syntetické snímky pomocou nášho vlastného generátora dát - starGen. Výkonnosť nášho modelu je vyhodnotená na reálnych dátach získaných z Astronomického a Geofyzikálneho observatória v Modre. Na zlepšenie výkonu tiež používame malé množstvo reálnych dát v kombinácii s technikami augmentácie na doladenie nášho modelu. V záverečnej časti porovnávame výkon nášho modelu s najmodernejšou sieťou ResNet-18.
